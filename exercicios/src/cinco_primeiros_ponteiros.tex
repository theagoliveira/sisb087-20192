\documentclass[a4paper,12pt]{article}
\usepackage[utf8]{inputenc}
\usepackage[T1]{fontenc}
\usepackage{enumitem}
\usepackage{amssymb}
\usepackage{amsmath}
\usepackage{fontawesome}
\usepackage[top=3cm,left=44mm,right=44mm,bottom=2cm]{geometry}
\usepackage[brazil]{babel}

\usepackage{sourcecodepro}
\renewcommand*\familydefault{\ttdefault}

\title{Linguagem C completa e descomplicada -- Cinco primeiros exercícios do capítulo 10}
\author{Prof. Thiago Cavalcante}
\date{}

\begin{document}

\pagenumbering{gobble}
\maketitle

\setcounter{section}{9}
\section{Ponteiros}

\setcounter{subsection}{6}
\subsection{Exercícios}

\begin{enumerate}
  \item Escreva um programa que contenha duas variáveis inteiras. Compare seus endereços e exiba o maior endereço.
  \item Escreva um programa que contenha duas variáveis inteiras. Leia essas variáveis do teclado. Em seguida, compare seus endereços e exiba o conteúdo do maior endereço.
  \item Crie um programa que contenha um array de \textbf{float} contendo 10 elementos. Imprima o endereço de cada posição desse array.
  \item Crie um programa que contenha uma matriz de \textbf{float} contendo três linhas e três colunas. Imprima o endereço de cada posição dessa matriz.
  \item Crie um programa que contenha um array de inteiros contendo cinco elementos. Utilizando apenas aritmética de ponteiros, leia esse array do teclado e imprima o dobro de cada valor lido.
\end{enumerate}

\end{document}
