\documentclass[a4paper,12pt]{article}
\usepackage[utf8]{inputenc}
\usepackage[T1]{fontenc}
\usepackage{enumitem}
\usepackage{amssymb}
\usepackage{amsmath}
\usepackage{fontawesome}
\usepackage[top=3cm,left=44mm,right=44mm,bottom=2cm]{geometry}
\usepackage[brazil]{babel}

\usepackage{sourcecodepro}
\renewcommand*\familydefault{\ttdefault}

\title{Linguagem C completa e descomplicada -- Cinco primeiros exercícios dos capítulos 6 ao 9}
\author{Prof. Thiago Cavalcante}
\date{}

\begin{document}

\pagenumbering{gobble}
\maketitle

\setcounter{section}{5}

\section{Vetores e matrizes – arrays}

\subsection{Exercícios}

\subsubsection{Vetores}

\begin{enumerate}
  \item Crie um programa que leia do teclado seis valores inteiros e em seguida mostra na tela os valores lidos.
  \item Crie um programa que leia do teclado seis valores inteiros e em seguida mostre na tela os valores lidos na ordem inversa.
  \item Faça um programa que leia cinco valores e os armazene em um vetor. Em seguida, mostre todos os valores lidos juntamente com a média dos valores.
  \item Faça um programa que possua um array de nome A que armazene seis números inteiros. O programa deve executar os seguintes passos:

  \begin{enumerate}
    \item Atribua os seguintes valores a esse array: 1, 0, 5, -2, -5, 7.
    \item Armazene em uma variável a soma dos valores das posições A[0], A[1] e A[5] do array e mostre na tela essa soma.
    \item Modifique o array na posição 4, atribuindo a essa posição o valor 100.
    \item Mostre na tela cada valor do array A, um em cada linha.
  \end{enumerate}

  \item Faça um programa que leia um vetor de oito posições. Em seguida, leia também dois valores X e Y quaisquer correspondentes a duas posições no vetor. Seu programa deverá exibir a soma dos valores encontrados nas respectivas posições X e Y.
\end{enumerate}

\subsubsection{Matrizes}

\begin{enumerate}
  \item Faça um programa que leia uma matriz de tamanho 3 $\times$ 3. Imprima na tela o menor valor contido nessa matriz.
  \item Faça um programa que leia uma matriz de tamanho 4 $\times$ 4. Imprima na tela o maior valor contido nessa matriz e a sua localização (linha e coluna).
  \item Faça um programa que declare uma matriz de tamanho 5 $\times$ 5. Preencha com 1 a diagonal principal e com 0 os demais elementos. Ao final, escreva a matriz obtida na tela.
  \item Leia uma matriz de tamanho 4 $\times$ 4. Em seguida, conte e escreva na tela quantos valores maiores do que 10 ela possui.
  \item Leia uma matriz de tamanho 4 $\times$ 4. Em seguida, conte e escreva na tela quantos valores negativos ela possui.
\end{enumerate}

\pagebreak
\section{Arrays de caracteres – strings}

\subsection{Exercícios}

\begin{enumerate}
  \item Faça um programa que leia uma string e a imprima na tela.
  \item Faça um programa que leia uma string e imprima as quatro primeiras letras dela.
  \item Sem usar a função strlen(), faça um programa que leia uma string e imprima quantos caracteres ela possui.
  \item Faça um programa que leia uma string e a imprima de trás para a frente.
  \item Faça um programa que leia uma string e a inverta. A string invertida deve ser armazenada na mesma variável. Em seguida, imprima a string invertida.
\end{enumerate}

\pagebreak
\section{Tipos definidos pelo programador}

\subsection{Exercícios}

\begin{enumerate}
  \item Implemente um programa que leia o nome, a idade e o endereço de uma pessoa e armazene esses dados em uma estrutura. Em seguida, imprima na tela os dados da estrutura lida.
  \item Crie uma estrutura para representar as coordenadas de um ponto no plano (posições X e Y). Em seguida, declare e leia do teclado um ponto e exiba a distância dele até a origem das coordenadas, isto é, a posição (0,0).
  \item Crie uma estrutura para representar as coordenadas de um ponto no plano (posições X e Y). Em seguida, declare e leia do teclado dois pontos e exiba a distância entre eles.
  \item Crie uma estrutura chamada Retângulo. Essa estrutura deverá conter o ponto superior esquerdo e o ponto inferior direito do retângulo. Cada ponto é definido por uma estrutura Ponto, a qual contém as posições X e Y. Faça um programa que declare e leia uma estrutura Retângulo e exiba a área e o comprimento da diagonal e o perímetro desse retângulo.
  \item Usando a estrutura Retângulo do exercício anterior, faça um programa que declare e leia uma estrutura Retângulo e um Ponto, e informe se esse ponto está ou não dentro do retângulo.
\end{enumerate}

\pagebreak
\section{Funções}

\subsection{Exercícios}

\subsubsection{Passagem por valor}

\begin{enumerate}
  \item Escreva uma função que receba por parâmetro dois números e retorne o maior deles.
  \item Faça uma função que receba um número inteiro de 1 a 12 e imprima em tela o mês de acordo com o número digitado pelo usuário. Exemplo: Entrada = 4. Saída = Abril.
  \item Escreva uma função que receba por parâmetro uma temperatura em graus Fahrenheit e a retorne convertida em graus Celsius. A fórmula de conversão é: $C = (F - 32.0) \times (5.0/9.0)$, sendo $F$ a temperatura em Fahrenheit e $C$ a temperatura em Celsius.
  \item Escreva uma função que receba por parâmetro a altura e o raio de um cilindro circular e retorne o volume desse cilindro. O volume de um cilindro circular é calculado por meio da seguinte fórmula: $V = \pi \times \textnormal{raio}^2 \times \textnormal{altura}$, em que $\pi = 3.141592$
  \item Escreva uma função para o cálculo do volume de uma esfera $V = 4 \times \pi \times r^3 / 3$, em que $\pi = 3.141592$ valor do raio $r$ deve ser passado por parâmetro.
\end{enumerate}

\subsubsection{Passagem por referência}

\begin{enumerate}
  \item Escreva uma função que, dado um número real passado como parâmetro, retorne a parte inteira e a parte fracionária desse número por referência.
  \item Escreva uma função para o cálculo do volume e área de uma esfera, $V = 4 \times \pi \times r^3 / 3$, $V = 4 \times \pi \times r^2$, em que $\pi = 3.141592$. O valor do raio $r$ deve ser passado por parâmetro, e os valores calculados devem ser retornados por referência.
  \item Escreva uma função que receba um array de 10 elementos e retorne a sua soma.
  \item Escreva uma função que receba um array contendo a nota de 10 alunos e retorne a média dos alunos.
  \item Escreva uma função que calcule o desvio-padrão $d$ de um vetor $V$ contendo $n$ números, em que $m$ é a média desse vetor.

  \begin{equation*}
    d = \sqrt{\frac{1}{n - 1} \sum_{i = 0}^{n - 1} (V[i] - m)^2}
  \end{equation*}
\end{enumerate}

\subsubsection{Recursão}

\begin{enumerate}
  \item Escreva uma função recursiva que calcule a soma dos primeiros $n$ cubos: $S = 1^3 + 2^3 + ... + n^3$
  \item Crie uma função recursiva que receba um número inteiro N e retorne o somatório dos números de 1 a N.
  \item Crie uma função recursiva que receba um número inteiro N e imprima todos os números naturais de 0 até N em ordem crescente.
  \item Crie uma função recursiva que receba um número inteiro N e imprima todos os números naturais de 0 até N em ordem decrescente.
  \item Crie uma função recursiva que retorne a soma dos elementos de um vetor de inteiros.
\end{enumerate}

\end{document}
