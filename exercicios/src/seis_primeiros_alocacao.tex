\documentclass[a4paper,10pt]{article}

\usepackage{appendixnumberbeamer}
\usepackage{booktabs}
\usepackage[scale=2]{ccicons}
\usepackage{pgfplots}
\usepackage{xspace}
\usepackage{bookmark}
\usepackage{amssymb}
\usepackage{mathtools}
\usepackage[normalem]{ulem}
\usepackage[T1]{fontenc}
\usepackage[sfdefault,book]{FiraSans}
\usepackage{FiraMono}
\usepackage{fontawesome}
\usepackage{hyperref}
\usepackage{listings}

\renewcommand{\thefootnote}{\fnsymbol{footnote}}
\renewcommand{\MintedPygmentize}{/home/thiago/.local/bin/pygmentize}

\definecolor{light-gray}{gray}{0.95}
\renewcommand\lstlistingname{Código}
\DeclareCaptionFormat{listing} {
  \parbox{\textwidth}{\hspace{-0.2cm}#1#2#3}
}
\DeclareCaptionFont{black}{\color{black}}
\captionsetup[lstlisting]{
  format=listing,
  labelfont=black,
  textfont=black,
  singlelinecheck=true,
  margin=0pt,
  font={tt,footnotesize,bf}
}

\lstset{
  basicstyle=\footnotesize\ttfamily,
  escapeinside={\%*}{*)},
  mathescape=true,
  showspaces=false,
  showtabs=false,
  showstringspaces=false,%
  rulesepcolor=\color{black},
  frame=shadowbox,
  upquote=true,
  literate=
  {á}{{\'a}}1 {é}{{\'e}}1 {í}{{\'i}}1 {ó}{{\'o}}1 {ú}{{\'u}}1
  {Á}{{\'A}}1 {É}{{\'E}}1 {Í}{{\'I}}1 {Ó}{{\'O}}1 {Ú}{{\'U}}1
  {à}{{\`a}}1 {è}{{\`e}}1 {ì}{{\`i}}1 {ò}{{\`o}}1 {ù}{{\`u}}1
  {À}{{\`A}}1 {È}{{\'E}}1 {Ì}{{\`I}}1 {Ò}{{\`O}}1 {Ù}{{\`U}}1
  {ä}{{\"a}}1 {ë}{{\"e}}1 {ï}{{\"i}}1 {ö}{{\"o}}1 {ü}{{\"u}}1
  {Ä}{{\"A}}1 {Ë}{{\"E}}1 {Ï}{{\"I}}1 {Ö}{{\"O}}1 {Ü}{{\"U}}1
  {â}{{\^a}}1 {ê}{{\^e}}1 {î}{{\^i}}1 {ô}{{\^o}}1 {û}{{\^u}}1
  {Â}{{\^A}}1 {Ê}{{\^E}}1 {Î}{{\^I}}1 {Ô}{{\^O}}1 {Û}{{\^U}}1
  {Ã}{{\~A}}1 {ã}{{\~a}}1 {Õ}{{\~O}}1 {õ}{{\~o}}1
  {œ}{{\oe}}1 {Œ}{{\OE}}1 {æ}{{\ae}}1 {Æ}{{\AE}}1 {ß}{{\ss}}1
  {ű}{{\H{u}}}1 {Ű}{{\H{U}}}1 {ő}{{\H{o}}}1 {Ő}{{\H{O}}}1
  {ç}{{\c c}}1 {Ç}{{\c C}}1 {ø}{{\o}}1 {å}{{\r a}}1 {Å}{{\r A}}1
  {€}{{\euro}}1 {£}{{\pounds}}1 {«}{{\guillemotleft}}1
  {»}{{\guillemotright}}1 {ñ}{{\~n}}1 {Ñ}{{\~N}}1 {¿}{{?`}}1
}

\setminted{
  fontsize=\footnotesize,
  style=bw,
  frame=single,
  labelposition=topline,
}

\pretitle{\begin{center}\normalsize\bfseries\MakeUppercase{Programação 2 -- }\MakeUppercase}
\author{\normalsize Prof. Thiago Cavalcante}
\date{\vspace{-4ex}}

\geometry{
  top=0.0cm,
  left=1.0cm,
  right=1.0cm,
  bottom=1.0cm
}

\setlength\columnsep{30pt}

\usetikzlibrary{arrows}

\makeatletter
\AddEnumerateCounter{\PaddingUp}{\two@digits}{A00}
\AddEnumerateCounter{\PaddingDown}{\two@digits}{A00}
\newcommand\PaddingUp[1]{\expandafter\two@digits\csname c@#1\endcsname}
\newcommand\PaddingDown[1]{\PaddingUp{#1}\addtocounter{#1}{-2}}
\makeatother

\makeatletter
\def\enumalphalphcnt#1{\expandafter\@enumalphalphcnt\csname c@#1\endcsname}
\def\@enumalphalphcnt#1{\AlphAlph{#1}}
\makeatother
\AddEnumerateCounter{\enumalphalphcnt}{\@enumalphalphcnt}{aa}

\pagenumbering{gobble}


\title{Linguagem C completa e descomplicada -- 6 primeiros exercícios do capítulo 11}

\begin{document}

\maketitle

\emergencystretch 3em

\begin{multicols*}{2}
\setcounter{section}{10}
\section{Alocação dinâmica}

\setcounter{subsection}{3}
\subsection{Exercícios}

\setlength{\leftmargini}{0pt}
\begin{enumerate}
  \item Escreva um programa que mostre o tamanho em byte que cada tipo de dados ocupa na memória: \textbf{char}, \textbf{int}, \textbf{float}, \textbf{double}.
  \item Crie uma estrutura representando um aluno de uma disciplina. Essa estrutura deve conter o número de matrícula do aluno, seu nome e as notas de três provas. Escreva um programa que mostre o tamanho em byte dessa estrutura.
  \item Crie uma estrutura chamada \textbf{Cadastro}. Essa estrutura deve conter o nome, a idade e o endereço de uma pessoa. Agora, escreva uma função que receba um inteiro positivo \textbf{N} e retorne o ponteiro para um vetor de tamanho \textbf{N}, alocado dinamicamente, dessa estrutura. Solicite também que o usuário digite os dados desse vetor dentro da função.
  \item Elabore um programa que leia do usuário o tamanho de um vetor a ser lido. Em seguida, faça a alocação dinâmica desse vetor. Por fim, leia o vetor do usuário e o imprima.
  \item Faça um programa que leia um valor inteiro N não negativo. Se o valor de \textbf{N} for inválido, o usuário deverá digitar outro até que ele seja válido (ou seja, positivo). Em seguida, leia um vetor \textbf{V} contendo \textbf{N} posições de inteiros, em que cada valor deverá ser maior ou igual a 2. Esse vetor deverá ser alocado dinamicamente.
  \item Escreva um programa que \textbf{aloque dinamicamente uma matriz} de inteiros. As dimensões da matriz deverão ser lidas do usuário. Em seguida, escreva uma função que receba um valor e retorne 1, caso o valor esteja na matriz, ou retorne 0, no caso contrário.
\end{enumerate}
\end{multicols*}
\end{document}
