\documentclass[a4paper,12pt]{article}
\usepackage[utf8]{inputenc}
\usepackage[T1]{fontenc}
\usepackage{enumitem}
\usepackage{amssymb}
\usepackage{amsmath}
\usepackage{fontawesome}
\usepackage[top=3cm,left=44mm,right=44mm,bottom=2cm]{geometry}
\usepackage[brazil]{babel}

\usepackage{sourcecodepro}
\renewcommand*\familydefault{\ttdefault}

\title{Linguagem C completa e descomplicada -- Seis primeiros exercícios do capítulo 11}
\author{Prof. Thiago Cavalcante}
\date{}

\begin{document}

\pagenumbering{gobble}
\maketitle

\sloppy
\raggedright

\setcounter{section}{10}
\section{Alocação dinâmica}

\setcounter{subsection}{3}
\subsection{Exercícios}

\begin{enumerate}
  \item Escreva um programa que mostre o tamanho em byte que cada tipo de dados ocupa na memória: \textbf{char}, \textbf{int}, \textbf{float}, \textbf{double}.
  \item Crie uma estrutura representando um aluno de uma disciplina. Essa estrutura deve conter o número de matrícula do aluno, seu nome e as notas de três provas. Escreva um programa que mostre o tamanho em byte dessa estrutura.
  \item Crie uma estrutura chamada \textbf{Cadastro}. Essa estrutura deve conter o nome, a idade e o endereço de uma pessoa. Agora, escreva uma função que receba um inteiro positivo \textbf{N} e retorne o ponteiro para um vetor de tamanho \textbf{N}, alocado dinamicamente, dessa estrutura. Solicite também que o usuário digite os dados desse vetor dentro da função.
  \item Elabore um programa que leia do usuário o tamanho de um vetor a ser lido. Em seguida, faça a alocação dinâmica desse vetor. Por fim, leia o vetor do usuário e o imprima.
  \item Faça um programa que leia um valor inteiro N não negativo. Se o valor de \textbf{N} for inválido, o usuário deverá digitar outro até que ele seja válido (ou seja, positivo). Em seguida, leia um vetor \textbf{V} contendo \textbf{N} posições de inteiros, em que cada valor deverá ser maior ou igual a 2. Esse vetor deverá ser alocado dinamicamente.
  \item Escreva um programa que \textbf{aloque dinamicamente uma matriz} de inteiros. As dimensões da matriz deverão ser lidas do usuário. Em seguida, escreva uma função que receba um valor e retorne 1, caso o valor esteja na matriz, ou retorne 0, no caso contrário.
\end{enumerate}

\end{document}
