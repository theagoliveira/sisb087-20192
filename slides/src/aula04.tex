\documentclass[10pt]{beamer}

\usepackage[utf8]{inputenc}
\usepackage[T1]{fontenc}
\usepackage{enumitem}
\usepackage{amssymb}
\usepackage{amsmath}
\usepackage{fontawesome}
\usepackage[top=3cm,left=44mm,right=44mm,bottom=2cm]{geometry}
\usepackage[brazil]{babel}
\usepackage{sourcecodepro}

\input{general/configs.tex}
\input{general/info.tex}

\subtitle{Aula 4}
\date{13 de novembro de 2019}

\begin{document}

\maketitle

\begin{frame}{Ponteiros}
  \huge
  \textbf{Ponteiros}
  \vfill
  \LARGE
  Tipo \textbf{especial} de variável que armazena um \textbf{endereço na memória} ao invés de um valor
  \vfill
  \large
  \setlength{\leftmargini}{0pt}
  \begin{itemize}
      \item [] Toda informação está armazenada na memória
      \item [] Ponteiros "apontam" para endereços de memória
      \item [] programador $\rightarrow$ nome, programa $\rightarrow$ endereço
  \end{itemize}
\end{frame}

\begin{frame}{Ponteiros}
  \huge
  \textbf{Ponteiros podem ser \alert{perigosos \faExclamationTriangle}}
  \vfill
  \large
  Podem apontar para um espaço na memória que está sendo usado para um outro propósito
\end{frame}

\begin{frame}[fragile]{Ponteiros}
  \huge
  \textbf{Declaração de ponteiro}
  \vfill
  \Large
  \begin{verbatim}
tipo_do_ponteiro *nome_do_ponteiro;
  \end{verbatim}
  \vfill
  \large
  Não confundir com o operador de multiplicação!
\end{frame}

\begin{frame}{Ponteiros}
  \huge
  \textbf{Inicialização e atribuição}
  \vfill
  \large
  \begin{itemize}
    \item \texttt{int *p;} (aponta para um \textbf{lugar indefinido} \faExclamationTriangle)
    \item \texttt{p = NULL;} (aponta para \textbf{lugar nenhum} -- diferente)
    \item \texttt{p = }\&\texttt{x;} (aponta para variável \texttt{x})
    \item \texttt{printf("\%d", *p)} (imprime o valor guardado na variável apontada por \texttt{p})
    \item \texttt{int *p2 = p;} (ponteiro \texttt{p2} aponta para o mesmo lugar que \texttt{p} \faExclamationTriangle)
  \end{itemize}
  \vfill
  \large
  \&: operador de endereçamento
\end{frame}

\begin{frame}{Ponteiros}
  \huge
  \textbf{Operadores}
  \vfill
  \Large
  \textbf{*}: Usado na \textbf{declaração} de um ponteiro e na \textbf{obtenção do conteúdo} para onde ele aponta (\alert{dois usos distintos})
  \vfill
  \Large
  \textbf{\&}: Usado para \textbf{obter o endereço} de uma variável
\end{frame}

\begin{frame}{Ponteiros}
  \huge
  \textbf{Aritmética com ponteiros}
  \vfill
  \Large
  \begin{itemize}
    \item Adição e subtração: incrementam ou decrementam o tamanho \textbf{em bytes} do tipo do ponteiro \faExclamationTriangle
    \item Outras operações: não são realizadas no ponteiro, mas no seu conteúdo (\textbf{\texttt{*p}})
  \end{itemize}

\end{frame}

\begin{frame}{Ponteiros}
  \huge
  Também é possível comparar ponteiros com operadores relacionais: \texttt{==}, \texttt{!=}, \texttt{>}, \texttt{<}, \texttt{>=}, \texttt{<=}

\end{frame}

\begin{frame}{Ponteiros}
  \huge
  \textbf{Ponteiros genéricos}
  \vfill
  \large
  \begin{itemize}
    \item \texttt{void *nome\_do\_ponteiro};
    \item Aponta para qualquer tipo de dado \faExclamationTriangle
    \item Acesso precisa ser feito com um \textit{typecast}: \texttt{*(int*)p}
    \item Incremento é feito de byte em byte
  \end{itemize}
\end{frame}

\begin{frame}[fragile]{Ponteiros}
  \huge
  \textbf{Ponteiros e arrays}
  \vfill
  \Large
  O nome de um array é apenas um \textbf{ponteiro para o primeiro elemento} do array
  \vfill
  \large
  \begin{lstlisting}
ponteiro = nome_array
ponteiro = %*\textnormal{\&}*)nome_array[0]

ponteiro[i] = 2;
*(ponteiro + i) = 2;
  \end{lstlisting}
\end{frame}

\begin{frame}[fragile]{Ponteiros}
  \huge
  \textbf{Array de ponteiros}
  \vfill
  \Large
  \begin{lstlisting}
tipo_de_dado *nome_do_array[tamanho];

nome_do_array[indice] = %*\textnormal{\&}*)variavel
  \end{lstlisting}
\end{frame}

\begin{frame}[fragile]{Ponteiros}
  \huge
  \textbf{Ponteiro para ponteiro}
  \vfill
  \Large
  \begin{verbatim}
tipo_de_dado **nome_do_ponteiro;
  \end{verbatim}
\end{frame}
\end{document}
