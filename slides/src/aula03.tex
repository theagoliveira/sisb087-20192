\documentclass[10pt]{beamer}

\usepackage[utf8]{inputenc}
\usepackage[T1]{fontenc}
\usepackage{enumitem}
\usepackage{amssymb}
\usepackage{amsmath}
\usepackage{fontawesome}
\usepackage[top=3cm,left=44mm,right=44mm,bottom=2cm]{geometry}
\usepackage[brazil]{babel}
\usepackage{sourcecodepro}

\input{general/configs.tex}
\input{general/info.tex}

\makeatletter
\newcommand\thefontsize[1]{{#1 The current font size is: \f@size pt\par}}
\makeatother

% \normalsize = 10.00pt
% \large      = 12.00pt
% \Large      = 14.40pt
% \LARGE      = 17.28pt
% \huge       = 20.74pt
% \Huge       = 24.88pt

\subtitle{Aula 3}
\date{06 de novembro de 2019}

\begin{document}

\maketitle

\begin{frame}{Funções}
  \huge
  \textbf{Bloco de código} que pode ser \textbf{nomeado} e \textbf{chamado} dentro de um programa
  \vfill
  \LARGE
  Exemplos: \texttt{scanf()} e \texttt{printf()}
\end{frame}

\begin{frame}{Funções}
  \huge
  \textbf{Por que usar funções?}
  \vfill
  \begin{itemize}
    \item Estruturação do programa
    \item Reutilização de código
  \end{itemize}
\end{frame}

\begin{frame}[fragile]{Funções}
  \huge
  \textbf{Declarando uma função}
  \vfill
  \large
  \begin{verbatim}
tipo_retornado nome_função(parâmetros) {
    declarações e comandos
}
  \end{verbatim}

  Nome da função segue regras de nome de variáveis
\end{frame}

\begin{frame}[fragile]{Funções}
  \huge
  \textbf{Local da declaração}
  \vfill
  \large
  \begin{enumerate}
    \item Declaração antes da \textbf{\texttt{main()}}
    \item Protótipo antes da \textbf{\texttt{main()}} e declaração depois
  \end{enumerate}
  \vfill
  \large
  O protótipo contém apenas o \textbf{cabeçalho} da função
\end{frame}

\begin{frame}{Funções}
  \huge
  \textbf{Funcionamento de uma função}
  \vfill
  \Large
  \begin{itemize}
    \item Fluxo do programa é interrompido
    \item Valores são copiados para parâmetros
    \item Comandos da função são executados
    \item Valor do \texttt{\textbf{return}} é copiado para variável
    \item Fluxo do programa continua
  \end{itemize}
\end{frame}

\begin{frame}[fragile]{Funções}
  \huge
  \textbf{Lista de parâmetros}
  \vfill
  \large
  \begin{verbatim}
(tipo1 nome1, tipo2 nome2, ..., tipoN nomeN)
  \end{verbatim}
  \vfill
  \setlength{\leftmargini}{0pt}
  \begin{itemize}
      \item [] o tipo deve vir antes de \textbf{cada} parâmetro, mesmo se os tipos forem iguais (ex.: \texttt{int x, int y})
      \item [] o parâmetro só pode ser acessado \textbf{dentro} da função
  \end{itemize}
\end{frame}

\begin{frame}[fragile]{Funções}
  \huge
  \textbf{A lista \alert{não é obrigatória}}
  \vfill
  \Large
  \begin{columns}
    \begin{column}{0.35\textwidth}
      \begin{lstlisting}
tipo nome%*\textbf{()}*) {
  ...
}
      \end{lstlisting}
    \end{column}
    \begin{column}{0.60\textwidth}
      \begin{lstlisting}
|
|  tipo nome%*\textbf{(void)}*) {
|    ...
|  }
|
      \end{lstlisting}
    \end{column}
  \end{columns}
  \vfill
  \large
  Existe uma diferença entre as declarações
\end{frame}

\begin{frame}{Funções}
  \huge
  \textbf{Corpo da função}
  \vfill
  \LARGE
  \begin{itemize}
    \item Sequência de declarações
    \item Sequência de comandos
  \end{itemize}
  \vfill
  \large
  \setlength{\leftmargini}{0pt}
  \begin{itemize}
      \item [] A \textbf{\texttt{main()}} é uma função presente cada programa
      \item [] Tudo o que é feito na \textbf{\texttt{main()}} pode ser feito em outras funções
  \end{itemize}
\end{frame}

\begin{frame}{Funções}
  \huge
  Em geral, \textbf{evita-se fazer leitura e escrita} de dados dentro de uma função
  \vfill
  Uma possível exceção: \textbf{menu de usuário}
\end{frame}

\begin{frame}[fragile]{Funções}
  \huge
  \textbf{Retorno da função}
  \vfill
  \Large
  \begin{verbatim}
    return expressão;
      \end{verbatim}
  \vfill
  \large
  Uma função pode retornar \textbf{qualquer um} dos tipos válidos em C (incluindo tipos definidos pelo usuário)
\end{frame}

\begin{frame}{Funções}
  \huge
  \textbf{Retorno da função}
  \vfill
  \Large
  Uma função pode \textbf{não retornar} nada também, basta definir o \texttt{tipo\_retornado} como \textbf{\texttt{void}}
  \vfill
  \large
  Exemplo: uma função para imprimir algo na tela
\end{frame}

\begin{frame}{Funções}
  \huge
  Uma função pode retornar:
  \vfill
  \Large
  \begin{itemize}
    \item Variável
    \item Constante
    \item Expressão artimética
    \item Expressão lógica
    \item Outra função
  \end{itemize}
  \vfill
  \large
  O retorno precisa ser compatível com o tipo definido
\end{frame}

\begin{frame}[fragile]{Funções}
  \huge
  Uma função pode ter \textbf{vários} comandos \texttt{return}
  \vfill
  \normalsize
  \begin{columns}
    \begin{column}{0.40\textwidth}
      \begin{lstlisting}

int max(int x, int y) {
  if(x > y)
    %*\textbf{return}*) x;
  else
    %*\textbf{return}*) y;
}

      \end{lstlisting}
    \end{column}
    \begin{column}{0.50\textwidth}
      \begin{lstlisting}
|  int max(int x, int y) {
|    int z;
|    if(x > y)
|      z = x;
|    else
|      z = y;
|    %*\textbf{return}*) z;
|  }
      \end{lstlisting}
    \end{column}
  \end{columns}
\end{frame}

\begin{frame}{Funções}
  \large
  \begin{itemize}
    \item A função \textbf{encerra} quando chega em um \textbf{\texttt{return}}
    \item Uma função do tipo \textbf{\texttt{void}} pode ser finalizada com \textbf{\texttt{return;}}
    \item Uma função \textbf{não pode} retornar um \textbf{array} (a não ser que esteja dentro de uma \texttt{struct})
  \end{itemize}
\end{frame}

\begin{frame}{Funções}
  \huge
  \textbf{Passagem de parâmetros}
  \vfill
  \begin{itemize}
    \item Por \textbf{valor}
    \item Por \textbf{referência}
  \end{itemize}
\end{frame}

\begin{frame}{Funções}
  \huge
  Passagem por \textbf{valor}
  \vfill
  \Large
  \begin{itemize}
    \item O argumento é \textbf{copiado} para o parâmetro
    \item O parâmetro é uma \textbf{variável local} da função
    \item Mudanças no parâmetro \textbf{não refletem} no argumento
    \item O parâmetro é \textbf{destruído} e o argumento mantém seu valor \textbf{original}
  \end{itemize}

\end{frame}

\begin{frame}[fragile]{Funções}
  \huge
  Passagem por \textbf{referência}
  \vfill
  \large
  \begin{itemize}
    \item Usada quando se quer \textbf{alterar} o valor do argumento
    \item Não é passado o valor da variável, mas sim o seu \textbf{endereço na memória}
    \item Na \textbf{declaração} e \textbf{corpo} da função, usa-se o operador \textbf{\texttt{*}} antes da variável
    \item Na \textbf{chamada} da função, usa-se o operador \textbf{\&} antes da variável
  \end{itemize}
  \vfill
  \large
  Exemplo: \texttt{scanf("\%d", }\&\texttt{x);}
\end{frame}

\begin{frame}{Funções}
  \huge
  \textbf{Passagem de arrays como parâmetros}
  \vfill
  \Large
  \begin{itemize}
    \item Arrays são \textbf{sempre} passados \textbf{por referência}
    \item É necessário sempre um segundo parâmetro contendo o \textbf{tamanho} do array
  \end{itemize}

\end{frame}

\begin{frame}{Funções}
  \huge
  \textbf{Declarando os parâmetros}
  \vfill
  \LARGE
  \begin{itemize}
    \item \texttt{int *array, int tamanho}
    \item \texttt{int array[], int tamanho}
    \item \texttt{int array[5], int tamanho}
  \end{itemize}
  \vfill
  \large
  Os primeiros elementos são todos equivalentes, pois o tamanho não é checado
\end{frame}

\begin{frame}[fragile]{Funções}
  \LARGE
  Não é necessário usar o operador \& na \textbf{chamada} da função para um array, pois o seu \textbf{nome representa o endereço do seu primeiro elemento} na memória
  \vfill
  \begin{lstlisting}
array = %*\textnormal{\&}*)array[0]
  \end{lstlisting}
\end{frame}

\begin{frame}[fragile]{Funções}
  \LARGE
  Não é necessário usar o operador \texttt{*} no \textbf{corpo} da função para um array. O \textbf{acesso aos seus elementos pode ser feito normalmente} com colchetes.
  \vfill
  \begin{lstlisting}
x = array[elemento];
  \end{lstlisting}
\end{frame}

\begin{frame}[fragile]{Funções}
  \huge
  \textbf{Arrays multidimensionais}
  \vfill
  \Large
  É necessário indicar o tamanho das dimensões extras na declaração da função.
  \vfill
  \begin{verbatim}
int array[][5], int tamanho
  \end{verbatim}

\end{frame}

\begin{frame}{Funções}
  \huge
  \textbf{Passagem de structs como parâmetros}
  \vfill
  \Large
  \begin{itemize}
    \item Passagem por \textbf{valor}

    \begin{itemize}
      \Large
      \item estrutura
      \item campo da estrutura
    \end{itemize}
    \item Passagem por \textbf{referência}

    \begin{itemize}
      \Large
      \item estrutura
      \item campo da estrutura
    \end{itemize}
  \end{itemize}
\end{frame}

\begin{frame}[fragile]{Funções}
  \huge
  \textbf{Operador seta}
  \vfill
  \large
  Quando a estrutura é passada por referência, o acesso a ela dentro da função pode ser feito com os operadores \textbf{\texttt{*}} e \textbf{.} em conjunto ou com o operador \textbf{\texttt{->}}
  \vfill
  \large
  \begin{verbatim}
(*struct).campo
struct->campo
  \end{verbatim}
\end{frame}

\begin{frame}{Funções}
  \huge
  \textbf{Recursão (definição circular)}
  \vfill
  \centering
  \includegraphics[height=.7\textheight]{images/recursao.jpg}
\end{frame}

\begin{frame}{Funções}
  \huge
  Uma \textit{funçao recursiva} \textbf{chama a si própria} dentro da sua definição.
\end{frame}

\begin{frame}{Funções}
  \Huge
  Exemplo clássico: \textbf{fatorial}
\end{frame}

\begin{frame}{Funções}
  \huge
  \textbf{Como funciona a recursividade}
  \vfill
  \LARGE
  \begin{itemize}
    \item Dividir e conquistar
    \item Caminho de ida $\rightarrow$ caso-base $\rightarrow$ caminho de volta
  \end{itemize}

\end{frame}

\begin{frame}{Funções}
  \huge
  \textbf{Cuidados a serem tomados}
  \vfill
  \LARGE
  \begin{itemize}
    \item Critério de parada
    \item Parâmetro da chamada recursiva
  \end{itemize}
\end{frame}

\begin{frame}{Funções}
  \LARGE
  Algoritmos recursivos são \textbf{considerados mais enxutos/elegantes}, porém \textbf{tendem a ser mais ineficientes} (tempo/memória) e apresentam maior dificuldade na detecção de erros
  \vfill
  Ex.: \textbf{Fibonacci}
\end{frame}

\end{document}
