\documentclass[10pt]{beamer}

\usepackage[utf8]{inputenc}
\usepackage[T1]{fontenc}
\usepackage{enumitem}
\usepackage{amssymb}
\usepackage{amsmath}
\usepackage{fontawesome}
\usepackage[top=3cm,left=44mm,right=44mm,bottom=2cm]{geometry}
\usepackage[brazil]{babel}
\usepackage{sourcecodepro}

\input{general/configs.tex}
\input{general/info.tex}

\subtitle{Aula 5}
\date{27 de novembro de 2019}

\begin{document}

\maketitle

\begin{frame}{Alocação dinâmica}
  \huge
  \textbf{Uma variável é uma \alert{posição na memória previamente* reservada} que pode armazenar um dado}
  \vfill
  *previamente $\rightarrow$ toda variável deve ser \alert{declarada}
\end{frame}

\begin{frame}{Alocação dinâmica}
  \huge
  Nem sempre é possível saber quanta memória um programa vai precisar (ex. funcionários)
\end{frame}

\begin{frame}{Alocação dinâmica}
  \huge
  \textbf{A linguagem C permite \alert{alocar dinamicamente} blocos de memórias usando ponteiros}
  \vfill
  \Large
  \setlength{\leftmargini}{0pt}
  \begin{itemize}
      \item [] Alocar $\rightarrow$ reservar
      \item [] Dinamicamente $\rightarrow$ em tempo de execução
  \end{itemize}
\end{frame}

\begin{frame}{Alocação dinâmica}
  \huge
  O programa requisita, \textbf{em tempo de execução}, um espaço de memória ao computador, o qual retorna um \textbf{endereço para o início} desse espaço (que é armazenado em um \textbf{ponteiro})
\end{frame}

\begin{frame}{Alocação dinâmica}
  \huge
  \textbf{Funções para alocação de memória (\texttt{stdlib.h})}
  \vfill
  \LARGE
  \begin{itemize}
    \item \texttt{sizeof}
    \item \texttt{malloc}
    \item \texttt{calloc}
    \item \texttt{realloc}
    \item \texttt{free}
  \end{itemize}
\end{frame}

\begin{frame}[fragile]{Alocação dinâmica}
  \huge
  \textbf{\texttt{sizeof}}
  \vfill
  Usada para saber quantos bytes ocupa \textbf{um único elemento} de um determinado tipo
  \vfill
  \large
  \begin{lstlisting}[language=C]
printf("%d", sizeof(int)); // resultado: 4
  \end{lstlisting}
\end{frame}

\begin{frame}[fragile]{Alocação dinâmica}
  \huge
  \textbf{\texttt{malloc}}
  \vfill
  \Large
  \begin{lstlisting}[language=C]
void *malloc (unsigned int num);
  \end{lstlisting}
  \vfill
  \large
  \setlength{\leftmargini}{0pt}
  \begin{itemize}
      \item [] Aloca uma quantidade \textbf{\texttt{num}} de bytes de memória e retorna um \textbf{ponteiro genérico (\texttt{void*})} para o início do bloco
      \item [] Programador \textbf{deve atribuir um tipo} ao ponteiro
      \item [] Se houver erro a função retorna \textbf{\texttt{NULL}}
  \end{itemize}
\end{frame}

\begin{frame}[fragile]{Alocação dinâmica}
  \large
  \begin{lstlisting}[language=C]
int i;
int *p;

p = (int *) malloc(5 * sizeof(int));

for(i = 0; i < 5; i++) {
  p[i] = 10 + i;
  // *(p + i) = 10 + i;
}
  \end{lstlisting}
  \vfill
  \large
  \faExclamationTriangle \quad Prestar atenção no tipo de dados e na quantidade
\end{frame}

\begin{frame}[fragile]{Alocação dinâmica}
  \huge
  \textbf{\texttt{calloc}}
  \vfill
  \Large
  \begin{lstlisting}[language=C]
void *calloc (unsigned int num,
              unsigned int size);
  \end{lstlisting}
  \vfill
  \large
  \setlength{\leftmargini}{0pt}
  \begin{itemize}
      \item [] Aloca uma quantidade $\textnormal{\textbf{\texttt{num}}} \times \textnormal{\textbf{\texttt{size}}}$ de bytes de memória e retorna um \textbf{ponteiro genérico (\texttt{void*})} para o início do bloco
      \item [] Inicializa todos os bits do bloco com \textbf{zero}
  \end{itemize}
\end{frame}

\begin{frame}[fragile]{Alocação dinâmica}
  \large
  \begin{lstlisting}[language=C]
int i;
int *p;

// p = (int *) malloc(5 * sizeof(int));
p = (int *) calloc(5, sizeof(int));

for(i = 0; i < 5; i++) {
  p[i] = 10 + i;
  // *(p + i) = 10 + i;
}
  \end{lstlisting}
\end{frame}

\begin{frame}[fragile]{Alocação dinâmica}
  \huge
  \textbf{\texttt{realloc}}
  \vfill
  \Large
  \begin{lstlisting}[language=C]
void *realloc (void *ptr,
               unsigned int num);
  \end{lstlisting}
  \vfill
  \large
  \setlength{\leftmargini}{0pt}
  \begin{itemize}
      \item [] Modifica para \textbf{\texttt{num}} bytes o tamanho da memória previamente alocada e apontada por \textbf{\texttt{ptr}}
  \end{itemize}
\end{frame}

\begin{frame}[fragile]{Alocação dinâmica}
  \large
  \vfill
  \begin{lstlisting}[language=C]
int *p;
p = malloc(5 * sizeof(int));

//Diminui o tamanho do array
p = realloc(p, 3 * sizeof(int));

//Aumenta o tamanho do array
p = realloc(p, 10 * sizeof(int));
  \end{lstlisting}
  \vfill
  Quando o tamanho é maior, o bloco recém-alocado tem valor \textbf{indeterminado} (não ocorre inicialização)
\end{frame}

\begin{frame}[fragile]{Alocação dinâmica}
  \huge
  \textbf{\texttt{free}}
  \vfill
  \Large
  \begin{lstlisting}[language=C]
void free (void *p);
  \end{lstlisting}
  \vfill
  \large
  \setlength{\leftmargini}{0pt}
  \begin{itemize}
      \item [] A memória alocada dinamicamente (\texttt{malloc}, \texttt{calloc} ou \texttt{realloc}) \textbf{não é liberada automaticamente} pelo programa \faExclamationTriangle
      \item [] Essa memória precisa ser liberada com a função \textbf{\texttt{free}}, que recebe o ponteiro para o início do bloco
  \end{itemize}
\end{frame}

\begin{frame}[fragile]{Alocação dinâmica}
  \huge
  \textbf{\alert{Sempre} libere a memória que não for mais utilizar e não deixe ponteiros "soltos"}
  \vfill
  \begin{lstlisting}[language=C]
free(p);
p = NULL;
  \end{lstlisting}
\end{frame}

\begin{frame}[fragile]{Alocação dinâmica}
  \huge
  \textbf{Arrays multidimensionais}
  \vfill
  \LARGE
  \begin{itemize}
    \item Usando array unidimensional
    \item Usando ponteiro para ponteiro
    \item Usando ponteiro para ponteiro + array unidimensional
  \end{itemize}
\end{frame}

\begin{frame}{Arquivos}
  \LARGE
  \textbf{Arquivos são coleções de bytes} armazenados em um dispositivo de armazenamento (como um disco rídigo, por exemplo)
  \vfill
  Um arquivo pode ser interpretado como um texto, uma imagem, um vídeo etc.
\end{frame}

\begin{frame}{Arquivos}
  \huge
  \textbf{Vantagens}
  \vfill
  \Large
  \begin{itemize}
    \item Armazenamento durável
    \item Grandes quantidades de informação
    \item Acesso sequencial ou não
    \item Acesso concorrente (vários programas)
  \end{itemize}
\end{frame}

\begin{frame}{Arquivos}
  \LARGE
  \centering
  \textbf{arquivos texto $\times$ arquivos binários}
\end{frame}

\begin{frame}{Arquivos}
  \huge
  \textbf{Arquivos texto}
  \vfill
  \Large
  \begin{itemize}
    \item Podem ser mostrados \textbf{diretamente na tela} ou modificados em um editor de texto como o \textbf{bloco de notas}
    \item Dados são gravados como \textbf{caracteres de 8 bits da tabela ASCII} (é feita uma conversão)
  \end{itemize}
\end{frame}

\begin{frame}{Arquivos}
  \huge
  \textbf{Arquivos binários}
  \vfill
  \Large
  \begin{itemize}
    \item Armazenam uma \textbf{sequência de bits} sujeita às convenções dos programas que a geraram
    \item São gravados \textbf{exatamente como estão organizados na memória}, sem conversão
    \item Exemplos: arquivos executáveis, arquivos compactados etc.
  \end{itemize}
\end{frame}

\begin{frame}{Arquivos}
  \huge
  \textbf{Leitura e escrita}
  \vfill
  \LARGE
  Funções da biblioteca \textbf{\texttt{stdio.h}}
  \vfill
  \begin{itemize}
    \item Abrir/fechar
    \item Leitura/escrita de caracteres/bytes
  \end{itemize}
\end{frame}

\begin{frame}[fragile]{Arquivos}
  \huge
  \textbf{Ponteiro para arquivo}
  \vfill
  \begin{lstlisting}[language=C]
FILE *p;
  \end{lstlisting}
  \vfill
  \large
  Aponta para uma área na memória chamada de \textbf{buffer}, que contém informações sobre o arquivo aberto
\end{frame}

\begin{frame}[fragile]{Arquivos}
  \huge
  \textbf{Abrindo um arquivo: \texttt{fopen}}
  \vfill
  \large
  \begin{lstlisting}[language=C]
FILE *fopen(char *nome_do_arquivo, char *modo)
  \end{lstlisting}
  \vfill
  \setlength{\leftmargini}{0pt}
  \begin{itemize}
      \item [] \textbf{\texttt{nome\_do\_arquivo}} pode ser um caminho \textbf{absoluto} ou \textbf{relativo}, com relação à localização do programa
      \item [] Retorna \textbf{\texttt{NULL}} caso ocorra um erro
  \end{itemize}
\end{frame}

\begin{frame}{Arquivos}
  \large
  \textbf{Localização do programa}

  \texttt{"C:\textbackslash Programas\textbackslash programa.c"}
  \vfill
  \textbf{Localização do arquivo (absoluta)}

  \texttt{"C:\textbackslash Programas\textbackslash Arquivos\textbackslash arquivo.txt"}
  \vfill
  \textbf{Localização do arquivo (relativa)}

  \texttt{".\textbackslash Arquivos\textbackslash arquivo.txt"}
\end{frame}

\begin{frame}{Arquivos}
  \footnotesize
  \begin{center}
    \begin{tabular}{@{}ll@{}}
      \toprule
      \textbf{Modo} & \textbf{Função} \\
      \midrule
      \texttt{"r"} & Leitura. Arquivo deve existir. \\
      \texttt{"w"} & Escrita. Cria arquivo, se não houver. Apaga o anterior, se ele existir. \\
      \texttt{"a"} & Escrita. Os dados serão adicionados no fim do arquivo (\textit{append}). \\
      \texttt{"r+"} & Leitura/escrita. O arquivo deve existir e pode ser modificado. \\
      \texttt{"w+"} & Leitura/escrita. Cria arquivo se não houver. Apaga o anterior se ele existir. \\
      \texttt{"a+"} & Leitura/escrita. Os dados serão adicionados no fim do arquivo (\textit{append}). \\
      \bottomrule
    \end{tabular}
  \end{center}
  \vfill
  \large
  Para arquivos binários, adiciona-se um \textbf{\texttt{b}} ao modo
\end{frame}

\begin{frame}[fragile]{Arquivos}
  \huge
  \textbf{Saindo de um programa em caso de erro}
  \vfill
  \Large
  \begin{lstlisting}[language=C]
void exit(int codigo_de_retorno)
  \end{lstlisting}
  \vfill
  \large
  Convenção: \texttt{\textbf{codigo\_de\_retorno}} é \texttt{\textbf{0}} para um término normal e outro número caso ocorra um problema
\end{frame}

\begin{frame}[fragile]{Arquivos}
  \huge
  \textbf{Fechando um arquivo: \texttt{fclose}}
  \vfill
  \large
  \begin{lstlisting}[language=C]
int fclose(FILE *fp)
  \end{lstlisting}
  \vfill
  \setlength{\leftmargini}{0pt}
  \begin{itemize}
      \item [] Retorna \texttt{\textbf{0}} no caso de sucesso e outro número, caso contrário
      \item [] Ao fechar um arquivo, todo caractere que tenha permanecido no buffer é \textbf{gravado} (também ocorre quando o buffer fica cheio)
      \item [] A função \texttt{\textbf{exit}} fecha todos os arquivos abertos
  \end{itemize}
\end{frame}

\begin{frame}[fragile]{Arquivos}
  \huge
  \textbf{Escrevendo um char: \texttt{fputc}}
  \vfill
  \large
  \begin{lstlisting}[language=C]
int fputc(int c, FILE *fp)
  \end{lstlisting}
  \vfill
  \setlength{\leftmargini}{0pt}
  \begin{itemize}
      \item [] Retorna a constante \textbf{\texttt{EOF}}, se houver erro na escrita, ou o \textbf{próprio caractere}, se ele foi escrito com sucesso
      \item [] Após a escrita, o \textbf{indicador de posição interna} é avançado em um caractere (pode ser visto como o \textbf{cursor})
      \item [] Pode-se escrever na tela com o arquivo \textbf{\texttt{stdout}}
  \end{itemize}
\end{frame}

\begin{frame}[fragile]{Arquivos}
  \huge
  \textbf{Lendo um char: \texttt{fgetc}}
  \vfill
  \large
  \begin{lstlisting}[language=C]
int fgetc(FILE *fp)
  \end{lstlisting}
  \vfill
  \setlength{\leftmargini}{0pt}
  \begin{itemize}
    \item [] Retorna a constante \textbf{\texttt{EOF}}, se houver erro na leitura, ou o \textbf{próprio caractere}, se ele foi lido com sucesso
    \item [] Após a leitura, o \textbf{indicador de posição interna} é avançado em um caractere
      \item [] Pode-se ler do teclado com o arquivo \textbf{\texttt{stdin}}
    \end{itemize}
\end{frame}

\begin{frame}[fragile]{Arquivos}
  \huge
  \textbf{Fim do arquivo: \texttt{feof}}
  \vfill
  \large
  \begin{lstlisting}[language=C]
int feof(FILE *fp)
  \end{lstlisting}
  \vfill
  \setlength{\leftmargini}{0pt}
  \begin{itemize}
      \item [] Retorna \texttt{\textbf{0}} caso o fim ainda não tenha sido atingido e outro número, caso contrário
  \end{itemize}
\end{frame}

\begin{frame}[fragile]{Arquivos}
  \huge
  \textbf{Posição atual: \texttt{ftell}}
  \vfill
  \large
  \begin{lstlisting}[language=C]
long int ftell(FILE *fp)
  \end{lstlisting}
  \vfill
  \setlength{\leftmargini}{0pt}
  \begin{itemize}
      \item [] Para \textbf{arquivos binários}, o valor retornado indica o \textbf{número de bytes lidos} a partir do início do arquivo
      \item [] Para arquivos \textbf{texto}, \textbf{não existe garantia} de que o valor retornado seja o número exato de bytes lidos a partir do início do arquivo (mas esse valor pode ser usado para se retornar ao mesmo ponto, de qualquer forma)
      \item [] Em caso de erro, retorna -1
  \end{itemize}
\end{frame}

\begin{frame}[fragile]{Arquivos}
  \huge
  \textbf{Escrevendo uma string: \texttt{fputs}}
  \vfill
  \large
  \begin{lstlisting}[language=C]
int fputs(char *str,FILE *fp)
  \end{lstlisting}
  \vfill
  \setlength{\leftmargini}{0pt}
  \begin{itemize}
      \item [] Retorna a constante \textbf{\texttt{EOF}}, se houver erro na escrita, ou um valor diferente de \texttt{\textbf{0}}, caso contrário
      \item [] Não insere uma nova linha \texttt{`\textbackslash n'} (responsabilidade do programador)
  \end{itemize}
\end{frame}

\begin{frame}[fragile]{Arquivos}
  \huge
  \textbf{Lendo uma string: \texttt{fgets}}
  \vfill
  \large
  \begin{lstlisting}[language=C]
char *fgets(char *str, int tamanho, FILE *fp)
  \end{lstlisting}
  \vfill
  \setlength{\leftmargini}{0pt}
  \begin{itemize}
      \item [] Retorna \textbf{\texttt{NULL}}, se houver erro na leitura, ou o ponteiro para o primeiro caractere da string, caso contrário
      \item [] Lê até encontrar uma nova linha \texttt{`\textbackslash n'} (  que fará parte da string) ou até receber \texttt{\textbf{tamanho - 1}} caracteres (o último é sempre \texttt{`\textbackslash 0'})
  \end{itemize}
\end{frame}

\begin{frame}[fragile]{Arquivos}
  \huge
  \textbf{Escrevendo blocos de bytes: \texttt{fwrite}}
  \vfill
  \large
  \begin{lstlisting}[language=C]
int fwrite(void *buffer, int nro_de_bytes,
           int count, FILE *fp)
  \end{lstlisting}
  \vfill
  \setlength{\leftmargini}{0pt}
  \begin{itemize}
      \item [] Retorna o número de unidades escritas com sucesso (pode ser menor que \textbf{\texttt{count}} \faExclamationTriangle)
  \end{itemize}
\end{frame}

\begin{frame}[fragile]{Arquivos}
  \huge
  \textbf{Lendo blocos de bytes: \texttt{fread}}
  \vfill
  \large
  \begin{lstlisting}[language=C]
int fread(void *buffer, int nro_de_bytes,
          int count, FILE *fp)
  \end{lstlisting}
  \vfill
  \setlength{\leftmargini}{0pt}
  \begin{itemize}
      \item [] Retorna o número de unidades escritas com sucesso (pode ser menor que \textbf{\texttt{count}} \faExclamationTriangle)
  \end{itemize}
\end{frame}

\begin{frame}{Arquivos}
  \huge
  \textbf{Boa prática: escrever no arquivo o tamanho da string/array antes do conteúdo (facilita na hora de ler, depois)}
\end{frame}

\begin{frame}[fragile]{Arquivos}
  \huge
  \textbf{Escrevendo dados formatados: \texttt{fprintf}}
  \vfill
  \large
  \begin{lstlisting}[language=C]
int fprintf(FILE *fp, "tipos de saída",
            lista de variáveis)
  \end{lstlisting}
  \vfill
  \setlength{\leftmargini}{0pt}
  \begin{itemize}
      \item [] Retorna o total de caracteres escritos, em caso de sucesso, ou um número negativo, caso contrário
  \end{itemize}
\end{frame}

\begin{frame}[fragile]{Arquivos}
  \huge
  \textbf{Lendo dados formatados: \texttt{fscanf}}
  \vfill
  \large
  \begin{lstlisting}[language=C]
int fscanf(FILE *fp, "tipos de entrada",
           lista de variáveis)
  \end{lstlisting}
  \vfill
  \setlength{\leftmargini}{0pt}
  \begin{itemize}
      \item [] Retorna o total de caracteres lidos, em caso de sucesso, ou a constante \textbf{\texttt{EOF}}, caso contrário
  \end{itemize}
\end{frame}

\begin{frame}{Arquivos}
  \huge
  \textbf{As funções \texttt{fprintf} e \texttt{fscanf} convertem os dados para caracteres, o que faz com que os arquivos sejam maiores e as operações levem mais tempo para serem executadas}
\end{frame}

\begin{frame}[fragile]{Arquivos}
  \huge
  \textbf{Movendo-se dentro do arquivo: \texttt{fseek}}
  \vfill
  \large
  \begin{lstlisting}[language=C]
int fseek(FILE *fp, long numbytes, int origem)
  \end{lstlisting}
  \vfill
  \setlength{\leftmargini}{0pt}
  \begin{itemize}
      \item [] Retorna \texttt{\textbf{0}}, caso a movimentação seja bem sucedida, ou outro valor, caso contrário
      \item [] \texttt{\textbf{origem}} pode ser \texttt{\textbf{SEEK\_SET}}, \texttt{\textbf{SEEK\_CUR}} ou \texttt{\textbf{SEEK\_END}}
      \item [] \texttt{\textbf{numbytes}} pode ser negativo
  \end{itemize}
\end{frame}

\begin{frame}[fragile]{Arquivos}
  \huge
  \textbf{Voltando ao início do arquivo: \texttt{rewind}}
  \vfill
  \large
  \begin{lstlisting}[language=C]
void rewind(FILE *fp)
  \end{lstlisting}
  \vfill
  \setlength{\leftmargini}{0pt}
  \begin{itemize}
      \item []
  \end{itemize}
\end{frame}

\begin{frame}[fragile]{Arquivos}
  \huge
  \textbf{Excluindo um arquivo: \texttt{remove}}
  \vfill
  \large
  \begin{lstlisting}[language=C]
int remove(char *nome_do_arquivo)
  \end{lstlisting}
  \vfill
  \setlength{\leftmargini}{0pt}
  \begin{itemize}
      \item [] Recebe o \textbf{caminho para o arquivo} e não o ponteiro para FILE
      \item [] Retorna \texttt{\textbf{0}}, caso o arquivo seja removido com sucesso, ou outro valor, caso contrário
  \end{itemize}
\end{frame}

\begin{frame}[fragile]{Arquivos}
  \huge
  \textbf{Erro ao acessar o arquivo: \texttt{ferror} e \texttt{perror}}
  \vfill
  \large
  \begin{lstlisting}[language=C]
int ferror(FILE *fp)
void perror(char *str)
  \end{lstlisting}
  \vfill
  \setlength{\leftmargini}{0pt}
  \begin{itemize}
      \item [] \texttt{ferror} detecta se a última operação realizada no arquivo produziu algum erro e retorna \texttt{\textbf{0}}, no caso negativo
      \item [] \texttt{perror} imprime na tela a string \textbf{\texttt{str}} seguida da mensagem de erro do sistema
  \end{itemize}
\end{frame}
\end{document}
