\documentclass[10pt]{beamer}

\usepackage[utf8]{inputenc}
\usepackage[T1]{fontenc}
\usepackage{enumitem}
\usepackage{amssymb}
\usepackage{amsmath}
\usepackage{fontawesome}
\usepackage[top=3cm,left=44mm,right=44mm,bottom=2cm]{geometry}
\usepackage[brazil]{babel}
\usepackage{sourcecodepro}

\input{general/configs.tex}
\input{general/info.tex}

\subtitle{Aula 5}
\date{27 de novembro de 2019}

\begin{document}

\maketitle

\begin{frame}{Alocação dinâmica}
  \huge
  \textbf{Uma variável é uma \alert{posição na memória previamente* reservada} que pode armazenar um dado}
  \vfill
  *previamente $\rightarrow$ toda variável deve ser \alert{declarada}
\end{frame}

\begin{frame}{Alocação dinâmica}
  \huge
  Nem sempre é possível saber quanta memória um programa vai precisar (ex. funcionários)
\end{frame}

\begin{frame}{Alocação dinâmica}
  \huge
  \textbf{A linguagem C permite \alert{alocar dinamicamente} blocos de memórias usando ponteiros}
  \vfill
  \Large
  \setlength{\leftmargini}{0pt}
  \begin{itemize}
      \item [] Alocar $\rightarrow$ reservar
      \item [] Dinamicamente $\rightarrow$ em tempo de execução
  \end{itemize}
\end{frame}

\begin{frame}{Alocação dinâmica}
  \huge
  O programa requisita, \textbf{em tempo de execução}, um espaço de memória ao computador, o qual retorna um \textbf{endereço para o início} desse espaço (que é armazenado em um \textbf{ponteiro})
\end{frame}

\begin{frame}{Alocação dinâmica}
  \huge
  \textbf{Funções para alocação de memória (\texttt{stdlib.h})}
  \vfill
  \LARGE
  \begin{itemize}
    \item \texttt{sizeof}
    \item \texttt{malloc}
    \item \texttt{calloc}
    \item \texttt{realloc}
    \item \texttt{free}
  \end{itemize}
\end{frame}

\begin{frame}[fragile]{Alocação dinâmica}
  \huge
  \textbf{\texttt{sizeof}}
  \vfill
  Usada para saber quantos bytes ocupa \textbf{um único elemento} de um determinado tipo
  \vfill
  \large
  \begin{lstlisting}[language=C]
printf("%d", sizeof(int)); // resultado: 4
  \end{lstlisting}
\end{frame}

\begin{frame}[fragile]{Alocação dinâmica}
  \huge
  \textbf{\texttt{malloc}}
  \vfill
  \Large
  \begin{lstlisting}[language=C]
void *malloc (unsigned int num);
  \end{lstlisting}
  \vfill
  \large
  \setlength{\leftmargini}{0pt}
  \begin{itemize}
      \item [] Aloca uma quantidade \textbf{\texttt{num}} de bytes de memória e retorna um \textbf{ponteiro genérico (\texttt{void*})} para o início do bloco
      \item [] Programador \textbf{deve atribuir um tipo} ao ponteiro
      \item [] Se houver erro a função retorna \textbf{\texttt{NULL}}
  \end{itemize}
\end{frame}

\begin{frame}[fragile]{Alocação dinâmica}
  \large
  \begin{lstlisting}[language=C]
int i;
int *p;

p = (int *) malloc(5 * sizeof(int));

for(i = 0; i < 5; i++) {
  p[i] = 10 + i;
  // *(p + i) = 10 + i;
}
  \end{lstlisting}
  \vfill
  \large
  \faExclamationTriangle \quad Prestar atenção no tipo de dados e na quantidade
\end{frame}

\begin{frame}[fragile]{Alocação dinâmica}
  \huge
  \textbf{\texttt{calloc}}
  \vfill
  \Large
  \begin{lstlisting}[language=C]
void *calloc (unsigned int num,
              unsigned int size);
  \end{lstlisting}
  \vfill
  \large
  \setlength{\leftmargini}{0pt}
  \begin{itemize}
      \item [] Aloca uma quantidade $\textnormal{\textbf{\texttt{num}}} \times \textnormal{\textbf{\texttt{size}}}$ de bytes de memória e retorna um \textbf{ponteiro genérico (\texttt{void*})} para o início do bloco
      \item [] Inicializa todos os bits do bloco com \textbf{zero}
  \end{itemize}
\end{frame}

\begin{frame}[fragile]{Alocação dinâmica}
  \large
  \begin{lstlisting}[language=C]
int i;
int *p;

// p = (int *) malloc(5 * sizeof(int));
p = (int *) calloc(5, sizeof(int));

for(i = 0; i < 5; i++) {
  p[i] = 10 + i;
  // *(p + i) = 10 + i;
}
  \end{lstlisting}
\end{frame}

\begin{frame}[fragile]{Alocação dinâmica}
  \huge
  \textbf{\texttt{realloc}}
  \vfill
  \Large
  \begin{lstlisting}[language=C]
void *realloc (void *ptr,
               unsigned int num);
  \end{lstlisting}
  \vfill
  \large
  \setlength{\leftmargini}{0pt}
  \begin{itemize}
      \item [] Modifica para \textbf{\texttt{num}} bytes o tamanho da memória previamente alocada e apontada por \textbf{\texttt{ptr}}
  \end{itemize}
\end{frame}

\begin{frame}[fragile]{Alocação dinâmica}
  \large
  \vfill
  \begin{lstlisting}[language=C]
int *p;
p = malloc(5 * sizeof(int));

//Diminui o tamanho do array
p = realloc(p, 3 * sizeof(int));

//Aumenta o tamanho do array
p = realloc(p, 10 * sizeof(int));
  \end{lstlisting}
  \vfill
  Quando o tamanho é maior, o bloco recém-alocado tem valor \textbf{indeterminado} (não ocorre inicialização)
\end{frame}

\begin{frame}[fragile]{Alocação dinâmica}
  \huge
  \textbf{\texttt{free}}
  \vfill
  \Large
  \begin{lstlisting}[language=C]
void free (void *p);
  \end{lstlisting}
  \vfill
  \large
  \setlength{\leftmargini}{0pt}
  \begin{itemize}
      \item [] A memória alocada dinamicamente (\texttt{malloc}, \texttt{calloc} ou \texttt{realloc}) \textbf{não é liberada automaticamente} pelo programa \faExclamationTriangle
      \item [] Essa memória precisa ser liberada com a função \textbf{\texttt{free}}, que recebe o ponteiro para o início do bloco
  \end{itemize}
\end{frame}

\begin{frame}[fragile]{Alocação dinâmica}
  \huge
  \textbf{\alert{Sempre} libere a memória que não for mais utilizar e não deixe ponteiros "soltos"}
  \vfill
  \begin{lstlisting}[language=C]
free(p);
p = NULL;
  \end{lstlisting}
\end{frame}

\begin{frame}[fragile]{Alocação dinâmica}
  \huge
  \textbf{Arrays multidimensionais}
  \vfill
  \LARGE
  \begin{itemize}
    \item Usando array unidimensional
    \item Usando ponteiro para ponteiro
    \item Usando ponteiro para ponteiro + array unidimensional
  \end{itemize}
\end{frame}
\end{document}
