\section{Introdução}

\subsection{Informações Gerais da Disciplina}

\begin{frame}[fragile]{Informações Gerais da Disciplina}
    \begin{center}
        \begin{tabular}{@{}ll@{}}
            \toprule
            \textbf{Nome} & Programação 2 \\
            \textbf{Código} & SISB087 \\
            \textbf{Semestre} & 3º \\
            \textbf{Carga Horária} & 72h \\
            \textbf{PPC} & 02/2019 \\
            \textbf{Turma} & 2019.2 \\
            \textbf{Horário} & Quarta, 19:00 -- 22:30 \\
            \textbf{Sala} & 5 \\
            \textbf{Lista de discussão} & Google Groups: \href{https://groups.google.com/forum/#!forum/sisb087_20192}{\alert{sisb087\_20192}} \\
            \textbf{Repositório} & GitHub: \href{https://github.com/theagoliveira/sisb087_20192}{\alert{theagoliveira/sisb087\_20192}} \\
            \bottomrule
        \end{tabular}
    \end{center}
\end{frame}

\subsection{Pesquisa}

\begin{frame}{Pesquisa}
    \setbeamertemplate{itemize/enumerate body begin}{\large}
    
    \begin{enumerate}
    \item Pode trazer um computador para aulas práticas?
    \item Já usou \textbf{Git}?
    \item Já usou \textbf{GitHub}?
    \item Consegue compreender textos em \textbf{inglês}?
    \end{enumerate}
    
    \bigskip
    \bigskip
    
    \centering
    \texttt{<NOME COMPLETO> <E-MAIL> <RESPOSTAS>}
\end{frame}

\subsection{Recapitulando: Programação 1}

\begin{frame}{Recapitulando: Programação 1}
    \begin{columns}
    \begin{column}{0.5\textwidth}
        \minipage[c][0.7\textheight][s]{\columnwidth}
            \begin{itemize}
                \item Linguagem: \textbf{C}
                \item Conceitos e técnicas de programação básica, procedimentos, algoritmos e programas
                \item Identificadores, constantes, variáveis e atribuição
                \item Tipos primitivos de dados
                \item Comandos de entrada e saída
                \item Operadores, funções e expressões
                \item Instruções condicionais e de repetição
            \end{itemize}
            \vfill
        \endminipage
    \end{column}
    \begin{column}{0.5\textwidth}
        \minipage[c][0.7\textheight][s]{\columnwidth}
            \begin{itemize}
                \item Tipos definidos pelo programador e tipos abstratos de dados
                \item Noções de ponteiros
                \item Estruturas compostas de dados: vetores, matrizes e registros
                \item Manipulação de uma cadeia de caracteres
                \item Noções de arquivos
                \item Programação de algoritmos usando uma LP estruturada
                \item Boas práticas de programação
            \end{itemize}
            \vfill
        \endminipage
    \end{column}
    \end{columns}
\end{frame}

\subsection{Ementa}

\begin{frame}{Ementa}
    \begin{itemize}
        \item Linguagem: \textbf{C}
        \item Importância das estruturas de dados na solução de problemas
        \only<1>{\item Vetores e matrizes}
        \only<2>{\item \sout{Vetores e matrizes}}
        \item Estruturas de dados lineares e não lineares
        \item Pilhas, filas, listas, árvores, florestas, introdução à grafos
        \item Implementação de estruturas de dados com alocação estática e dinâmica de memória
        \item Implementação de estruturas de dados com e sem ponteiros
        \item Algoritmos de ordenação
        \item Algoritmos de busca
        \item Programação avançada e resolução de problemas complexos
        \item Introdução à análise de algoritmos
    \end{itemize}
\end{frame}

\subsection{Bibliografia}

\begin{frame}{Bibliografia}
    \huge
    Livros sugeridos no \alert{repositório}
    
    \Large
    \begin{itemize}
        \item Programação em C
        \only<1>{\item Estruturas de dados com exemplos em C}
        \only<2>{\item \alert{Estruturas de dados com exemplos em C}}
        \item Estruturas de dados com exemplos em outras linguagens
    \end{itemize}
\end{frame}

\subsection{Cronograma}

\begin{frame}{Cronograma}
    \begin{center}
        \begin{tabular}{@{}r@{/}r@{/}rll@{}}
            \toprule
            \multicolumn{4}{@{}l}{\textbf{Datas importantes}} \\
            \midrule
            13 & 11 & 2019 & \textit{Possível} dia da \alert{AB1} \\
            20 & 11 & 2019 & Feriado \\
            23 & 11 & 2019 & Prazo final para digitação da \alert{AB1} \\
            12 & 02 & 2020 & \textit{Possível} dia da \alert{AB2} \\
            17 & 02 & 2020 & Prazo final para digitação da \alert{AB2} \\
            17--22 & 02 & 2020 & Período de \alert{reavaliação} \\ 
            27--29 & 02 & 2020 & Período de \alert{provas finais} \\ 
            \bottomrule
        \end{tabular}
    \end{center}
\end{frame}

\section{Estruturas de Dados}

\setbeamertemplate{frame footer}{SCHILDT, H. C Completo e Total}
\begin{frame}{Introdução a Estruturas de Dados}
    \Large
    
    \onslide<1->{
    \begin{center}
        \setlength{\fboxsep}{2\fboxsep}\boxed{\textbf{$\text{Programa} = \text{Algoritmos} + \text{Estruturas de Dados}$}}
    \end{center}
    }
    
    \bigskip
    
    \onslide<2->{
    Escolha e implementação da estrutura de dados \textbf{são tão importantes quanto} as rotinas
    }
    
    \bigskip
    
    \onslide<3->{
    O tipo de estrutura é \textbf{determinado pela natureza} do problema
    }
\end{frame}
\setbeamertemplate{frame footer}{}

\setbeamertemplate{frame footer}{SKIENA, S. S. The Algorithm Design Manual}
\begin{frame}{Introdução a Estruturas de Dados}
    \Large

    \onslide<1->{
    \textit{Tipos abstratos de dados} podem ser implementados \textbf{corretamente} com diferentes estruturas de dados
    }
    
    \bigskip
    
    \onslide<2->{
    A forma como as operações são realizadas por uma estrutura podem \textbf{melhorar} (ou \textbf{piorar}) drasticamente a performance de um programa
    }
    
    \bigskip
    
    \onslide<3->{
    Projeto: melhor prevenir do que remediar
    }
\end{frame}
\setbeamertemplate{frame footer}{}

\setbeamertemplate{frame footer}{LOUDON, K. Mastering Algorithms with C}
\begin{frame}{Introdução a Estruturas de Dados}
    \Large

    Três motivos para usar estruturas de dados:
    
    \large
    
    \begin{itemize}
        \setlength\itemsep{1em}
        \onslide<1->{\item \textbf{Eficiência}: a forma de organização dos dados pode aumentar a \alert{velocidade de execução} de algoritmos (de busca e/ou ordenação, por exemplo)}
        \onslide<2->{\item \textbf{Abstração}: melhor \alert{entendimento} dos dados na solução de problemas}
        \onslide<3->{\item \textbf{Reusabilidade}: estruturas de dados são \alert{modulares} e \alert{independentes de contexto}}
    \end{itemize}
\end{frame}
\setbeamertemplate{frame footer}{}

\begin{frame}{Introdução a Estruturas de Dados}
    \Large

    Características:
    
    \begin{itemize}
        \onslide<1->{\item Estrutura Linear x Não-linear}
        \onslide<2->{\item Estrutura Contígua x Ligada}
        \onslide<3->{\item Dados Homogêneos x Heterogêneos}
        \onslide<4->{\item Alocação de Memória Estática x Dinâmica}
    \end{itemize}
\end{frame}

\setbeamertemplate{frame footer}{ZIVIANI, N. Projetos de Algoritmos Com Implementações em Pascal e C}
\begin{frame}{Tipos Abstratos de Dados -- TAD}
    \Large

    "Um TAD pode ser visto como um modelo matemático, acompanhado das operações definidas sobre o modelo."
\end{frame}
\setbeamertemplate{frame footer}{}

\setbeamertemplate{frame footer}{SEDGEWICK, R. Algorithms in C, Parts 1-4: Fundamentals, Data Structures, Sorting, Searching}
\begin{frame}{Tipos Abstratos de Dados -- TAD}
    \Large

    "Um TAD é um tipo de dados (conjunto de valores e operações sobre esses valores) que é acessado apenas através de uma interface."
\end{frame}
\setbeamertemplate{frame footer}{}

\setbeamertemplate{frame footer}{CELES, W.; CERQUEIRA, R.; RANGEL, J. L. Introdução a Estruturas de Dados}
\begin{frame}{Tipos Abstratos de Dados -- TAD}
    \Large

    "Nos casos em que um módulo define um novo tipo de dado e o conjunto de operações para manipular dados desse tipo, dizemos que o módulo representa um TAD. Nesse contexto, abstrato significa 'esquecida a forma de implementação', ou seja, um TAD é descrito pela finalidade do tipo e de suas operações, e não pela forma como está implementado."
\end{frame}
\setbeamertemplate{frame footer}{}

\begin{frame}{Tipos Abstratos de Dados -- TAD}
    \Large

    Exemplos:
    
    \begin{itemize}
        \item TAD Ponto
        \item TAD Círculo
    \end{itemize}
\end{frame}

\begin{frame}[fragile]{Exercício: TAD Matriz}
    \begin{verbatim}
typedef struct matriz Matriz;
        
Matriz* mat_cria(int m, int n);
void mat_libera(Matriz* mat);
float mat_acessa(Matriz* mat, int i, int j);
void mat_atribui(Matriz* mat, int i, int j, float v);
int mat_linhas(Matriz* mat);
int mat_colunas(Matriz* mat);

struct matriz {
    int lin;
    int col;
    float* v;
    /* outra opção: matriz como vetor de ponteiros */
    /* float** v; */
}
    \end{verbatim}
\end{frame}

\begin{frame}[standout]{}
    \huge
    Dúvidas?
\end{frame}
