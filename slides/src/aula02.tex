\documentclass[10pt]{beamer}

\usepackage[utf8]{inputenc}
\usepackage[T1]{fontenc}
\usepackage{enumitem}
\usepackage{amssymb}
\usepackage{amsmath}
\usepackage{fontawesome}
\usepackage[top=3cm,left=44mm,right=44mm,bottom=2cm]{geometry}
\usepackage[brazil]{babel}
\usepackage{sourcecodepro}

\input{general/configs.tex}
\input{general/info.tex}

\subtitle{Aula 2}
\date{30 de outubro de 2019}

\begin{document}

\maketitle

\begin{frame}{Vetores e matrizes (Arrays)}
    \huge
    \textbf{Estrutura de dados}

    \begin{itemize}
        \item Linear
        \item Contígua
        \item Homogênea
        \item Estática
    \end{itemize}
\end{frame}

\begin{frame}{Vetores e matrizes (Arrays)}
    \huge \textbf{múltiplas} variáveis $\times$ \textbf{um} array
\end{frame}

\begin{frame}[fragile]{Vetores e matrizes (Arrays)}
    \begin{verbatim}
#include <stdio.h>
#include <stdlib.h>

int main() {
    float n1, n2, n3;
    printf("Digite a nota de 3 estudantes: ");
    scanf("%f",&n1);
    scanf("%f",&n2);
    scanf("%f",&n3);
    float media = (n1 + n2 + n3) / 3.0;
    if(n1 > media) printf("nota: %f\n", n1);
    if(n2 > media) printf("nota: %f\n", n2);
    if(n3 > media) printf("nota: %f\n", n3);
    return 0;
}
    \end{verbatim}
\end{frame}

\begin{frame}[fragile]{Vetores e matrizes (Arrays)}
    \begin{verbatim}
#include <stdio.h>
#include <stdlib.h>

int main() {
    float n1, n2, n3, n4, n5, n6, n7, n8, n9, n10, n11, n12;
    printf("Digite a nota de 100 estudantes: ");
    scanf("%f",&n1);
    scanf("%f",&n2);
    scanf("%f",&n3);
    scanf("%f",&n4);
    scanf("%f",&n5);
    scanf("%f",&n6);
    scanf("%f",&n7);
    scanf("%f",&n8);
    scanf("%f",&n9);
    scanf("%f",&n10);
    scanf("%f",&n11);
    scanf("%f",&n12);
}
    \end{verbatim}
\end{frame}

\begin{frame}[fragile]{Vetores e matrizes (Arrays)}
    \huge
    \textbf{Declarando um array}

    \bigskip

    \begin{verbatim}
tipo nome[tamanho];

float notas[100];
    \end{verbatim}

    \Large
    tamanho é um valor inteiro e não pode ser uma variável
\end{frame}

\begin{frame}[fragile]{Vetores e matrizes (Arrays)}
    \huge
    \textbf{Acessando um elemento}

    \bigskip

    \begin{verbatim}
nome[índice]
    \end{verbatim}

    \Large
    \setlength{\leftmargini}{0pt}
    \begin{itemize}
        \item [] $0 \leq \text{índice} \leq (\text{tamanho} - 1)$ \faExclamationTriangle
        \item [] acesso funciona como uma variável
        \item [] índice informa quantas posições pular na memória
    \end{itemize}
\end{frame}

\begin{frame}[fragile]{Vetores e matrizes (Arrays)}
    \begin{verbatim}
#include <stdio.h>
#include <stdlib.h>

int main() {
    int notas[100];
    int i;
    for (i = 0; i < 100; i++){
        printf("Digite a nota do aluno %d", i);
        scanf("%d", &notas[i]);
    }
    return 0;
}
    \end{verbatim}
\end{frame}

\begin{frame}[fragile]{Vetores e matrizes (Arrays)}
    \huge
    \textbf{Arrays multidimensionais}

    \bigskip

    \Large
    \begin{verbatim}
tipo nome[tam_1][tam_2]...[tam_n];

int arr[100][200][50];
    \end{verbatim}


    2 dimensões $\Longrightarrow$ matriz
\end{frame}

\begin{frame}[fragile]{Vetores e matrizes (Arrays)}
    \huge
    \textbf{Acessando um elemento}

    \bigskip

    \Large
    \begin{verbatim}
nome[ind_1][ind_2]...[ind_n]
    \end{verbatim}

    \setlength{\leftmargini}{0pt}
    \begin{itemize}
        \item [] $0 \leq \text{ind\_1} \leq (\text{tam\_1} - 1)$
        \item [] $0 \leq \text{ind\_2} \leq (\text{tam\_2} - 1)$
        \item [] ...
        \item [] $0 \leq \text{ind\_3} \leq (\text{tam\_n} - 1)$
    \end{itemize}
\end{frame}

\begin{frame}{Vetores e matrizes (Arrays)}
    \huge
    arrays são \textbf{sempre} armazenados linearmente na memória
\end{frame}

\begin{frame}[fragile]{Vetores e matrizes (Arrays)}
    \huge
    \textbf{Inicializando um array}

    \begin{itemize}
        \item Laço \textbf{for}
        \item Atribuição direta
        \item Operador chaves $\{ \}$

        \begin{itemize}
            \Large
            \item Com/sem tamanho
            \item Total/parcial
        \end{itemize}
    \end{itemize}

    \Large

    não se pode atribuir um array a outro \faExclamationTriangle
\end{frame}

\begin{frame}[fragile]{Vetores e matrizes (Arrays)}
    \begin{verbatim}
int i; /* LAÇO FOR */
int arr[3];
for(i = 0; i < 3; i++) {
    arr[i] = 2 * i;
}

int arr[3]; /* ATRIBUIÇÃO DIRETA */
arr[0] = 0;
arr[1] = 2;
arr[2] = 4;

int arr[3] = {0, 2, 4}; /* OPERADOR CHAVES */
int arr[]  = {0, 2, 4}; // sem tamanho
int arr[3] = {0, 2};    // parcial
    \end{verbatim}
\end{frame}

\begin{frame}[fragile]{Vetores e matrizes (Arrays)}
    \begin{verbatim}
int i, k; /* LAÇO FOR */
int arr[2][2];
for(i = 0; i < 2; i++) {
    for(j = 0; j < 2; j++) {
        arr[i][j] = i + j;
    }
}

int arr[2][2]; /* ATRIBUIÇÃO DIRETA */
arr[0][0] = 0;
arr[0][1] = 1;
arr[1][0] = 1;
arr[1][1] = 2;

int arr[2][2] = {0, 1, 1, 2}; /* OPERADOR CHAVES */
int arr[2][2] = {{0, 1}, {1, 2}};
    \end{verbatim}
\end{frame}

\begin{frame}{Vetores e matrizes (Arrays)}
    \huge \textbf{Exercícios}
\end{frame}

\begin{frame}[fragile]{Arrays de caracteres (Strings)}
    \huge
    \textbf{Declarada como um array comum do tipo char}

    \bigskip

    \begin{verbatim}
char nome[tamanho];
    \end{verbatim}
\end{frame}

\begin{frame}{Arrays de caracteres (Strings)}
    \huge
    \textbf{Principal diferença}

    \bigskip

    caractere \texttt{'\textbackslash 0'} \textbf{obrigatório} para indicar o final da string

    \vfill

    \Large
    string de tamanho $x$ armazena $x - 1$ caracteres
\end{frame}

\begin{frame}[fragile]{Arrays de caracteres (Strings)}
    \huge
    \textbf{Inicializando uma string}

    \bigskip

    \large
    \begin{verbatim}
char str[20] = {'U', 'F', 'A', 'L', '\0'};
char str[20] = "UFAL";
    \end{verbatim}

    \vfill

    \Large
    não se pode atribuir uma string a outra \faExclamationTriangle
\end{frame}

\begin{frame}[fragile]{Arrays de caracteres (Strings)}
    \begin{verbatim}
/* COPIANDO UMA STRING */

#include <stdio.h>
#include <stdlib.h>

int main() {
    int i;
    char str1[20] = "Hello World";
    char str2[20];
    for (i = 0; str1[i] != '\0'; i++)
        str2[i] = str1[i];
    str2[i] = '\0';
    return 0;
}
    \end{verbatim}
\end{frame}

\begin{frame}[fragile]{Arrays de caracteres (Strings)}
    \huge \textbf{Lendo uma string do teclado}

    \bigskip

    \Large
    \begin{verbatim}
char string[20];

scanf("%s", string);
gets(string);
fgets(string, 20, stdin);
    \end{verbatim}
\end{frame}

\begin{frame}{Arrays de caracteres (Strings)}
    \Large
    \begin{itemize}
        \item [\faThumbsODown] \texttt{scanf}: desconsidera espaços
        \item [\faThumbsODown] \texttt{gets}: não evita estouro do \textit{buffer}
        \item [\faThumbsOUp] \texttt{fgets}: mais recomendada, não desconsidera espaços, impede o estouro do \textit{buffer} e leva em conta o \texttt{'\textbackslash n'}
    \end{itemize}
\end{frame}

\begin{frame}[fragile]{Arrays de caracteres (Strings)}
    \huge \textbf{Escrevendo uma string na tela}

    \bigskip

    \Large
    \begin{verbatim}
char string[20] = "UFAL";

printf("%s", string);
fputs(string, stdout);
    \end{verbatim}
\end{frame}

\begin{frame}[fragile]{Arrays de caracteres (Strings)}
    \huge \textbf{Funções para manipulação de strings}

    \bigskip

    \begin{verbatim}
#include <string.h>
    \end{verbatim}
\end{frame}

\begin{frame}[fragile]{Arrays de caracteres (Strings)}
    \Large
    \begin{itemize}
        \item \texttt{strlen(str)}

        \begin{itemize}
            \large
            \item retorna tamanho da string
            \item não considera o \texttt{'\textbackslash 0'}
        \end{itemize}

        \item \texttt{strcpy(dest, orig)}

        \begin{itemize}
            \large
            \item copia uma string em outra \faExclamationTriangle
        \end{itemize}

        \item \texttt{strcat(dest, orig)}

        \begin{itemize}
            \large
            \item insere uma string no final de outra \faExclamationTriangle
        \end{itemize}

        \item \texttt{strcomp(str1, str2)}

        \begin{itemize}
            \large
            \item checa se duas strings são iguais
            \item \textit{case-sensitive}
            \item \texttt{0}: iguais / \texttt{1}: diferentes
        \end{itemize}
    \end{itemize}
\end{frame}

\begin{frame}{Arrays de caracteres (Strings)}
    \huge \textbf{Exercícios}
\end{frame}

\begin{frame}{Tipos definidos pelo programador}
    \huge
    \textbf{Tipos básicos}: \texttt{char}, \texttt{int}, \texttt{float}, \texttt{double}, \texttt{void}

    \bigskip

    \textbf{Tipos compostos}: \texttt{array}
\end{frame}

\begin{frame}{Tipos definidos pelo programador}
    \huge \textbf{Comandos usados para definir novos tipos de dados}

    \begin{itemize}
        \item \texttt{struct} (estruturas/registros)
        \item \texttt{union} (uniões)
        \item \texttt{enum} (enumerações)
        \item \texttt{typedef}
    \end{itemize}
\end{frame}

\begin{frame}[fragile]{Tipos definidos pelo programador}
    \huge
    \textbf{Declarando uma estrutura}

    \bigskip

    \large
    \begin{columns}
    \begin{column}{0.35\textwidth}
        \begin{verbatim}
struct nome {
    tipo1 campo1;
    tipo2 campo2;
    ...
    tipoN campoN;
};
        \end{verbatim}
    \end{column}
    \begin{column}{0.55\textwidth}
        \begin{verbatim}
| struct cadastro {
|     char nome[50];
|     int  idade;
|     char rua[100];
|     int  numero;
| };
        \end{verbatim}
    \end{column}
    \end{columns}

    \vfill

    \setlength{\leftmargini}{0pt}
    \begin{itemize}
        \item [] variável que agrupa variáveis
        \item [] estruturas diferentes podem ter campos iguais
    \end{itemize}
\end{frame}

\begin{frame}[fragile]{Tipos definidos pelo programador}
    \huge
    \textbf{Declarando uma variável do tipo da estrutura}

    \bigskip

    \large
    \begin{verbatim}
struct nome {
    tipo1 campo1;
    tipo2 campo2;
    ...
    tipoN campoN;
} variavel1, variavel2;

struct nome variavel3, variavel4;
    \end{verbatim}

\end{frame}

\begin{frame}[fragile]{Tipos definidos pelo programador}
    \huge
    \textbf{Acessando os campos de uma estrutura (operador ponto \texttt{'.'})}

    \bigskip

    \large
    \begin{verbatim}
struct cadastro c;

strcpy(c.nome, "Thiago");
c.idade = 473;
strcpy(c.rua, "Rua Principal");
c.numero = 1082;
    \end{verbatim}

    \vfill

    cada campo pode ser visto como uma variável comum
\end{frame}

\begin{frame}[fragile]{Tipos definidos pelo programador}
    \huge
    \textbf{Inicialização de estruturas}

    \large
    \begin{verbatim}
/* TOTAL */
struct cadastro c = {
    "Thiago",
    473,
    "Rua Principal",
    1082
};
/* PARCIAL */
struct cadastro c = {"Thiago", 473};
    \end{verbatim}
\end{frame}

\begin{frame}[fragile]{Tipos definidos pelo programador}
    \huge
    \textbf{Array de estruturas}

    \bigskip

    \Large
    \begin{verbatim}
struct cadastro c1, c2, c3, c4;

struct cadastro c[4];
c[2].idade = 200;
    \end{verbatim}
\end{frame}

\begin{frame}[fragile]{Tipos definidos pelo programador}
    \huge
    \textbf{Atribuição entre estruturas}

    \bigskip

    \large
    \begin{columns}
    \begin{column}{0.35\textwidth}
        \begin{verbatim}
struct ponto {
    int x;
    int y;
};
        \end{verbatim}
    \end{column}
    \begin{column}{0.55\textwidth}
        \begin{verbatim}
| struct outro_ponto {
|    int x;
|    int y;
| };
        \end{verbatim}
    \end{column}
    \end{columns}

    \begin{verbatim}
struct ponto p1, p2 = {1, 2};
struct outro_ponto p3 = {3, 4};

p1 = p2; /* OK */
p1 = p3; /* ERRADO */
    \end{verbatim}
\end{frame}

\begin{frame}[fragile]{Tipos definidos pelo programador}
    \huge
    \textbf{Estruturas aninhadas}

    \large
    \begin{columns}
    \begin{column}{0.35\textwidth}
        \begin{verbatim}
struct endereco {
    char rua[50];
    int  numero;
};

        \end{verbatim}
    \end{column}
    \begin{column}{0.55\textwidth}
        \begin{verbatim}
| struct cadastro {
|     char nome[50];
|     int  idade;
|     struct endereco end;
| };
        \end{verbatim}
    \end{column}
    \end{columns}

    \begin{verbatim}
struct cadastro cad;
cad.end.numero = 10;
    \end{verbatim}

\end{frame}

\begin{frame}[fragile]{Tipos definidos pelo programador}
    \huge
    \textbf{Declarando uma união}

    \Large
    \begin{verbatim}
union nome {
    tipo1 campo1;
    tipo2 campo2;
    ...
    tipoN campoN;
}
    \end{verbatim}
\end{frame}

\begin{frame}[fragile]{Tipos definidos pelo programador}
    \huge
    \textbf{\texttt{struct} $\times$ \texttt{union}}

    \vfill

    \large
    \setlength{\leftmargini}{0pt}
    \begin{itemize}
        \item [] estrutura reserva espaço para armazenar todos os campos
        \item [] união reserva espaço para o maior campo e a memória é compartilhada
        \item [] apenas um membro pode ser armazenado por vez
        \item [] a modificação de um elemento afeta todos
    \end{itemize}

\end{frame}

\begin{frame}[fragile]{Tipos definidos pelo programador}
    \large
    \begin{verbatim}
union exemplo {
    short int x;     /* TAMANHO: 2 BYTES */
    unsigned char c; /* TAMANHO: 1 BYTE */
}

union exemplo u;     /* TAMANHO: 2 BYTES */
    \end{verbatim}
\end{frame}

\begin{frame}[fragile]{Tipos definidos pelo programador}
    \huge
    \textbf{Principal uso da união}

    \bigskip

    \large
    \begin{verbatim}
union exemplo {
  short int x;        /* TAMANHO: 2 BYTES */
  unsigned char c[2]; /* TAMANHO: 2 BYTES */
}
    \end{verbatim}

    subdividir um tipo básico em um array de partes menores
\end{frame}

\begin{frame}[fragile]{Tipos definidos pelo programador}
    \huge
    \textbf{Declarando uma enumeração}

    \bigskip

    \large
    \begin{verbatim}
enum nome { identificadores };
enum semana {
    Domingo, Segunda, Terca, Quarta,
    Quinta, Sexta, Sabado
};
    \end{verbatim}

    \large
    \setlength{\leftmargini}{0pt}
    \begin{itemize}
        \item [] identificadores são palavras separadas por vírgula
        \item [] palavras constituem as constantes criadas pela enumeração
    \end{itemize}
\end{frame}


\begin{frame}[fragile]{Tipos definidos pelo programador}
    \huge
    \textbf{Declarando uma variável do tipo da enumeração}

    \bigskip

    \large
    \begin{verbatim}
enum semana {
    Domingo, Segunda, Terca, Quarta,
    Quinta, Sexta, Sabado
} s1 s2;

enum semana s3;
    \end{verbatim}
\end{frame}

\begin{frame}[fragile]{Tipos definidos pelo programador}
    \huge
    Enumeração pode ser vista como uma \textbf{lista de constantes}

    \bigskip

    Cada constante é representada por um \textbf{número inteiro}, começando do \texttt{0}

\end{frame}

\begin{frame}[fragile]{Tipos definidos pelo programador}
    \huge
    \textbf{Valores das constantes podem ser definidos pelo programador}

    \large
    \begin{verbatim}
enum semana {
    Domingo=1,
    Segunda, /* =  2 */
    Terca,   /* =  3 */
    Quarta=7,
    Quinta,  /* =  8 */
    Sexta,   /* =  9 */
    Sabado   /* = 10 */
}; /* VALORES DA TABELA ASCII SÃO VÁLIDOS */
    \end{verbatim}
\end{frame}

\begin{frame}[fragile]{Tipos definidos pelo programador}
    \huge \textbf{Comando \texttt{typedef}}

    \bigskip

    \Large
    \begin{verbatim}
typedef tipo_existente novo_nome;
typedef int inteiro;
    \end{verbatim}

    \large
    \setlength{\leftmargini}{0pt}
    \begin{itemize}
        \item [] não cria um novo tipo, apenas um sinônimo para um tipo existente
        \item [] variáveis \texttt{tipo\_existente} e \texttt{novo\_nome} podem ser usadas conjuntamente (são do mesmo tipo)
    \end{itemize}
\end{frame}

\begin{frame}[fragile]{Tipos definidos pelo programador}
    \huge \textbf{\texttt{typedef}$+$\texttt{struct}}

    \bigskip

    \large
    \begin{columns}
    \begin{column}{0.40\textwidth}
        \begin{verbatim}
typedef struct cad {
  char nome[50];
  int  idade;
  char rua[100];
  int  numero;
} Cadastro;
        \end{verbatim}
    \end{column}
    \begin{column}{0.50\textwidth}
        \begin{verbatim}
| typedef struct {
|   char nome[50];
|   int  idade;
|   char rua[100];
|   int  numero;
| } Cadastro;
        \end{verbatim}
    \end{column}
    \end{columns}

    \begin{verbatim}
typedef struct cad Cadastro;
    \end{verbatim}
\end{frame}

\begin{frame}[fragile]{Tipos definidos pelo programador}
    \huge \textbf{Exercícios}
\end{frame}

\end{document}
