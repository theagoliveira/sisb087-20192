\documentclass[a4paper,10pt]{article}

\usepackage{appendixnumberbeamer}
\usepackage{booktabs}
\usepackage[scale=2]{ccicons}
\usepackage{pgfplots}
\usepackage{xspace}
\usepackage{bookmark}
\usepackage{amssymb}
\usepackage{mathtools}
\usepackage[normalem]{ulem}
\usepackage[T1]{fontenc}
\usepackage[sfdefault,book]{FiraSans}
\usepackage{FiraMono}
\usepackage{fontawesome}
\usepackage{hyperref}
\usepackage{listings}

\renewcommand{\thefootnote}{\fnsymbol{footnote}}
\renewcommand{\MintedPygmentize}{/home/thiago/.local/bin/pygmentize}

\definecolor{light-gray}{gray}{0.95}
\renewcommand\lstlistingname{Código}
\DeclareCaptionFormat{listing} {
  \parbox{\textwidth}{\hspace{-0.2cm}#1#2#3}
}
\DeclareCaptionFont{black}{\color{black}}
\captionsetup[lstlisting]{
  format=listing,
  labelfont=black,
  textfont=black,
  singlelinecheck=true,
  margin=0pt,
  font={tt,footnotesize,bf}
}

\lstset{
  basicstyle=\footnotesize\ttfamily,
  escapeinside={\%*}{*)},
  mathescape=true,
  showspaces=false,
  showtabs=false,
  showstringspaces=false,%
  rulesepcolor=\color{black},
  frame=shadowbox,
  upquote=true,
  literate=
  {á}{{\'a}}1 {é}{{\'e}}1 {í}{{\'i}}1 {ó}{{\'o}}1 {ú}{{\'u}}1
  {Á}{{\'A}}1 {É}{{\'E}}1 {Í}{{\'I}}1 {Ó}{{\'O}}1 {Ú}{{\'U}}1
  {à}{{\`a}}1 {è}{{\`e}}1 {ì}{{\`i}}1 {ò}{{\`o}}1 {ù}{{\`u}}1
  {À}{{\`A}}1 {È}{{\'E}}1 {Ì}{{\`I}}1 {Ò}{{\`O}}1 {Ù}{{\`U}}1
  {ä}{{\"a}}1 {ë}{{\"e}}1 {ï}{{\"i}}1 {ö}{{\"o}}1 {ü}{{\"u}}1
  {Ä}{{\"A}}1 {Ë}{{\"E}}1 {Ï}{{\"I}}1 {Ö}{{\"O}}1 {Ü}{{\"U}}1
  {â}{{\^a}}1 {ê}{{\^e}}1 {î}{{\^i}}1 {ô}{{\^o}}1 {û}{{\^u}}1
  {Â}{{\^A}}1 {Ê}{{\^E}}1 {Î}{{\^I}}1 {Ô}{{\^O}}1 {Û}{{\^U}}1
  {Ã}{{\~A}}1 {ã}{{\~a}}1 {Õ}{{\~O}}1 {õ}{{\~o}}1
  {œ}{{\oe}}1 {Œ}{{\OE}}1 {æ}{{\ae}}1 {Æ}{{\AE}}1 {ß}{{\ss}}1
  {ű}{{\H{u}}}1 {Ű}{{\H{U}}}1 {ő}{{\H{o}}}1 {Ő}{{\H{O}}}1
  {ç}{{\c c}}1 {Ç}{{\c C}}1 {ø}{{\o}}1 {å}{{\r a}}1 {Å}{{\r A}}1
  {€}{{\euro}}1 {£}{{\pounds}}1 {«}{{\guillemotleft}}1
  {»}{{\guillemotright}}1 {ñ}{{\~n}}1 {Ñ}{{\~N}}1 {¿}{{?`}}1
}

\setminted{
  fontsize=\footnotesize,
  style=bw,
  frame=single,
  labelposition=topline,
}

\pretitle{\begin{center}\normalsize\bfseries\MakeUppercase{Programação 2 -- }\MakeUppercase}
\author{\normalsize Prof. Thiago Cavalcante}
\date{\vspace{-4ex}}

\geometry{
  top=0.0cm,
  left=1.0cm,
  right=1.0cm,
  bottom=1.0cm
}

\setlength\columnsep{30pt}

\usetikzlibrary{arrows}

\makeatletter
\AddEnumerateCounter{\PaddingUp}{\two@digits}{A00}
\AddEnumerateCounter{\PaddingDown}{\two@digits}{A00}
\newcommand\PaddingUp[1]{\expandafter\two@digits\csname c@#1\endcsname}
\newcommand\PaddingDown[1]{\PaddingUp{#1}\addtocounter{#1}{-2}}
\makeatother

\makeatletter
\def\enumalphalphcnt#1{\expandafter\@enumalphalphcnt\csname c@#1\endcsname}
\def\@enumalphalphcnt#1{\AlphAlph{#1}}
\makeatother
\AddEnumerateCounter{\enumalphalphcnt}{\@enumalphalphcnt}{aa}

\pagenumbering{gobble}


\title{Prova 2}

\begin{document}

\maketitle

\emergencystretch 3em

\begin{itemize}[itemsep=0em]
  \item Não use celular/computador e não converse com ninguém, a prova é individual.
  \item Sinta-se à vontade para tirar dúvidas (\textbf{razoáveis}) ou pedir esclarecimentos sobre as questões.
  \item Use \textbf{letra legível}! não posso dar nota para algo que não consigo ler.
  \item Lembre-se de \textbf{assinar seu nome nas suas folhas}. Se usar \textbf{mais de uma} folha, \textbf{enumere cada página}.
  \item \textbf{Seja organizado:} especifique número e letra da questão que você está respondendo e deixe um espaço entre as respostas, para não ficar tudo amontoado. Você pode pegar mais folhas, se precisar.
\end{itemize}


NOME: \rule{.85\textwidth}{0.1mm}

\begin{multicols*}{2}
\setlength{\leftmargini}{0pt}
\begin{enumerate}
  \item (2,8 pt $\rightarrow$ 7 x 0,4 pt) Preencha os espaços:

  \begin{enumerate}
    % \item A função \rule{1cm}{0.2mm} grava uma string em um arquivo. % fputs
    \item A função \rule{1cm}{0.2mm} é usada na liberação da memória alocada dinamicamente para um elemento de uma lista encadeada. % free
    \item A função \rule{1cm}{0.2mm} move a posição atual de leitura/escrita no arquivo para o início. % rewind
    \item A função \rule{1cm}{0.2mm} grava um caractere em um arquivo. % fputc
    \item A função \rule{1cm}{0.2mm} lê uma string de um arquivo. % fgets
    % \item A função \rule{1cm}{0.2mm} lê dados de um arquivo da mesma forma que a função \texttt{scanf} lê do teclado. % fscanf
    % \item A função \rule{1cm}{0.2mm} move a posição atual de leitura/escrita no arquivo para um ponto específico. % fseek
    % \item A função \rule{1cm}{0.2mm} fecha um arquivo. % fclose
    \item A função \rule{1cm}{0.2mm} é usada para alocar dinamicamente um novo elemento de uma lista encadeada. % malloc
    \item A função \rule{1cm}{0.2mm} geralmente é usada para gravar blocos de bytes em um arquivo binário. % fwrite
    \item A função \rule{1cm}{0.2mm} abre um arquivo. % fopen
    % \item A função \rule{1cm}{0.2mm} lê um caractere de um arquivo. % fgetc
    % \item A função \rule{1cm}{0.2mm} normalmente é usada quando se lê blocos de bytes de um arquivo binário. % fread
  \end{enumerate}

  \item (2,1 pt $\rightarrow$ 3 x 0,7 pt) Verdadeiro ou falso. Justifique sua resposta para afirmações falsas.

  \begin{enumerate}
    % \item Se a posição atual de leitura/escrita no arquivo não estiver no início, o arquivo precisa ser fechado e reaberto para que a leitura possa ser feita a partir do início. % Falso. A função rewind pode ser usada para mover a posição para o início.
    \item A função \texttt{fprintf} pode imprimir os dados na tela. % Verdadeiro
    \item A função \texttt{fscanf} não pode ser usada para ler dados do teclado. % Falso. Pode ser usada para ler dados do teclado usando o ponteiro para o arquivo stdin.
    \item A função \texttt{fseek} só pode fazer a busca a partir do início de um arquivo. % Falso. Pode fazer a busca a partir do início, do fim ou da posição atual.
  \end{enumerate}

  \item (2,0 pt $\rightarrow$ 2 x 1,0 pt) Escreva o código para completar as tarefas a seguir:
  % FILE* arq1;
  % FILE* arq2;
  %
  % arq1 = fopen("arquivo1.txt", "r");
  % arq2 = fopen("arquivo2.txt", "a");
  %
  % primeiro_char = fgetc(arq1);
  % fprintf("int %d", sizeof(int));

  \begin{minted}{c}
#include <stdio.h>

int main () {
  char primeiro_char;
  // << SEU CÓDIGO ENTRA AQUI >>
  fclose(arq1);
  fclose(arq2);
}
  \end{minted}

  \begin{enumerate}
    \item Abra um arquivo chamado \texttt{"arquivo1.txt"}, em uma variável chamada \texttt{arq1}, no modo de leitura. Leia o primeiro caractere do arquivo em uma variável chamada \texttt{primeiro\_char}.
    % \item Abra um arquivo chamado \texttt{arquivo2.txt}, em uma variável chamada \texttt{arq2}, no modo de escrita em arquivo binário, de forma que o arquivo seja criado, caso não exista, ou apagado e escrito por cima, caso já exista.
    \item Abra um arquivo chamado \texttt{"arquivo2.txt"}, em uma variável chamada \texttt{arq2}, no modo de escrita, de forma que os dados sejam adicionados ao final do arquivo. Escreva no arquivo, com a função \texttt{fprintf}, a string \texttt{"int"} seguida de um espaço e o tamanho em bytes de um inteiro. Use a função \texttt{sizeof} para obter esse tamanho.
  \end{enumerate}

  \bigskip

  \item (3,1 pt) Leia o código abaixo e faça as tarefas:

  \begin{minted}{c}
#include <stdio.h>
#include <stdlib.h>

// DECLARAÇÃO DO ELEMENTO DA LISTA
struct lista {
  int dados;
  struct lista* prox;
};

// SINÔNIMO PARA STRUCT LISTA
typedef struct lista Lista;

// FUNÇÕES concatena_listas E media_lista
Lista* concatena_listas(Lista* l1, Lista* l2) {
  Lista* p;
  // << SEU CÓDIGO ENTRA AQUI >>
  return l1;
}

float media_lista(Lista* l) {
  Lista* p;
  int soma = 0;
  int contador = 0;
  // << SEU CÓDIGO ENTRA AQUI >>
  return (float) soma/contador;
}

int main () {
  int i;
  Lista* lst1, lst2;

  lst1 = cria_lista();
  lst2 = cria_lista();

  for(i = 1; i < 4; i++) {
    lst1 = insere_lista(lst1, i);
    lst2 = insere_lista(lst2, i + 6);
  }

  imprime_lista(lst1);
  printf("\n%f\n", media_lista(lst1));

  imprime_lista(lst2);
  printf("\n%f\n", media_lista(lst2));

  lst1 = concatena_listas(lst1, lst2);

  imprime_lista(lst1);
  printf("\n%f\n", media_lista(lst1));
}
  \end{minted}

  \begin{enumerate}
    \item (1,3 pt) Complete a função \texttt{concatena\_listas}, que deve inserir a lista \texttt{l2} no final da lista \texttt{l1}.
    % Lista* concatena(Lista* l1, Lista* l2) {
    %   Lista* p = l1;
    %
    %   while (p->prox != NULL) {
    %     p = p->prox;
    %   }
    %
    %   p->prox = l2;
    %
    %   return l1;
    % }
    \item (1,0 pt) Complete a função \texttt{media\_lista}, que calcula a média dos elementos em uma lista.
    % float media_lista(Lista* l) {
    %   Lista* p;
    %   int soma = 0;
    %   int contador = 0;
    %
    %   for(p = l; p != NULL; p = p->prox) {
    %     soma += p->dados;
    %     contador++;
    %   }
    %
    %   return (float) soma/contador;
    % }
    \item (0,8 pt) Escreva o que vai ser impresso na tela com a execução do programa.
    % 3 2 1
    % 2
    % 9 8 7
    % 8
    % 3 2 1 9 8 7
    % 5
  \end{enumerate}
\end{enumerate}
\end{multicols*}
\end{document}
