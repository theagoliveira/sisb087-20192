\documentclass[a4paper,10pt]{article}

\usepackage[utf8]{inputenc}
\usepackage[T1]{fontenc}
\usepackage{enumitem}
\usepackage{amssymb}
\usepackage{amsmath}
\usepackage{fontawesome}
\usepackage[top=3cm,left=44mm,right=44mm,bottom=2cm]{geometry}
\usepackage[brazil]{babel}
\usepackage{sourcecodepro}

\input{configs.tex}

\title{Prova 2 -- Questão 4}

\begin{document}

\maketitle

\emergencystretch 3em

\input{recommendations.tex}

NOME: \rule{.85\textwidth}{0.1mm}

\begin{multicols*}{2}
\setlength{\leftmargini}{0pt}
\begin{enumerate}
  \item [4.] (3,1 pt) Leia o código abaixo e faça as tarefas:

  \begin{minted}{c}
#include <stdio.h>
#include <stdlib.h>

// DECLARAÇÃO DO ELEMENTO DA LISTA
struct lista {
  int dados;
  struct lista* prox;
};

// SINÔNIMO PARA STRUCT LISTA
typedef struct lista Lista;

// FUNÇÕES concatena_listas E media_lista
Lista* concatena_listas(Lista* l1, Lista* l2) {
  Lista* p;
  // << SEU CÓDIGO ENTRA AQUI >>
  return l1;
}

float media_lista(Lista* l) {
  Lista* p;
  int soma = 0;
  int contador = 0;
  // << SEU CÓDIGO ENTRA AQUI >>
  return (float) soma/contador;
}

int main () {
  int i;
  Lista* lst1, lst2;

  lst1 = cria_lista();
  lst2 = cria_lista();

  for(i = 1; i < 4; i++) {
    lst1 = insere_lista(lst1, i);
    lst2 = insere_lista(lst2, i + 6);
  }

  imprime_lista(lst1);
  printf("\n%f\n", media_lista(lst1));

  imprime_lista(lst2);
  printf("\n%f\n", media_lista(lst2));

  lst1 = concatena_listas(lst1, lst2);

  imprime_lista(lst1);
  printf("\n%f\n", media_lista(lst1));
}
  \end{minted}

  \vfill\null
  \columnbreak

  \begin{enumerate}
    \item (1,3 pt) Complete a função \texttt{concatena\_listas}, que deve inserir a lista \texttt{l2} no final da lista \texttt{l1}.

    \begin{minted}{c}
Lista* concatena(Lista* l1, Lista* l2) {
  Lista* p;
  // RESPOSTA ////////////////////////////////
  for(p = l1; p->prox != NULL; p = p->prox);
  p->prox = l2;
  ////////////////////////////////////////////
  return l1;
}
    \end{minted}

    \item (1,0 pt) Complete a função \texttt{media\_lista}, que calcula a média dos elementos em uma lista.

    \begin{minted}{c}
float media_lista(Lista* l) {
  Lista* p;
  int soma = 0;
  int contador = 0;
  // RESPOSTA ////////////////////////////////
  for(p = l; p != NULL; p = p->prox) {
    soma += p->dados;
    contador++;
  }
  ////////////////////////////////////////////
  return (float) soma/contador;
}
    \end{minted}

    \item (0,8 pt) Escreva o que vai ser impresso na tela com a execução do programa.

    \begin{minted}{c}
3 2 1
2.00
9 8 7
8.00
3 2 1 9 8 7
5.00
    \end{minted}
  \end{enumerate}
\end{enumerate}
\end{multicols*}
\end{document}
