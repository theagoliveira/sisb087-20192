\documentclass[a4paper,10pt]{article}

\usepackage[utf8]{inputenc}
\usepackage[T1]{fontenc}
\usepackage{enumitem}
\usepackage{amssymb}
\usepackage{amsmath}
\usepackage{fontawesome}
\usepackage[top=3cm,left=44mm,right=44mm,bottom=2cm]{geometry}
\usepackage[brazil]{babel}
\usepackage{sourcecodepro}

\input{configs.tex}

\title{Final}

\begin{document}

\maketitle

\emergencystretch 3em

\input{recommendations.tex}

NOME: \rule{.85\textwidth}{0.1mm}

\begin{multicols*}{2}
\setlength{\leftmargini}{0pt}
\begin{enumerate}
  \item (3,5 pt $\rightarrow$ 7 x 0,5 pt) Preencha os espaços a seguir.

  \begin{enumerate}
    \item O comando \rule{1cm}{0.2mm} em uma função é usado para passar o valor de uma expressão de volta ao ponto do programa onde a função foi chamada. % return
    \item A função \rule{1cm}{0.2mm} normalmente é usada quando se lê blocos de bytes de um arquivo binário. % fread
    \item Uma função que chama a si mesma é uma função \rule{1cm}{0.2mm}. % recursiva
    \item Os nomes dos cinco primeiros elementos de um array \texttt{p} são \rule{1cm}{0.2mm}, \rule{1cm}{0.2mm}, \rule{1cm}{0.2mm}, \rule{1cm}{0.2mm} e \rule{1cm}{0.2mm}. % p[0], p[1], p[2], p[3], p[4]
    \item A função \rule{1cm}{0.2mm} escreve uma string em um arquivo. % fputs
    \item Um elemento de uma lista encadeada guarda informações e um \rule{1cm}{0.2mm} para o próximo elemento da lista. % ponteiro
    \item Uma variável de ponteiro contém como valor o \rule{1cm}{0.2mm} de outra variável. % endereço
  \end{enumerate}

  \item (2,8 pt $\rightarrow$ 4 x 0,7 pt) Verdadeiro ou falso. Justifique sua resposta para afirmações falsas.

  \begin{enumerate}
    \item O operador \texttt{\&} retorna o local na memória em que seu operando está armazenado % Verdadeiro.
    \item A função \texttt{fprintf} não pode imprimir os dados na tela. % Falso. Pode ser usada para ler dados do teclado usando o ponteiro para o arquivo stdout.
    \item Um array pode armazenar muitos tipos diferentes de valores. % Falso. Um array armazena apenas um tipo de valor.
    \item Os campos de diferentes estruturas podem ter nomes iguais. % Verdadeiro.
  \end{enumerate}

  \item (1,0 pt) Escreva uma função que recebe um número inteiro e retorna:

  \begin{itemize}
    \item \textcolor{white}{-}0, se ele for igual a zero
    \item \textcolor{white}{-}1, se ele for par
    \item -1, se ele for ímpar
  \end{itemize}
  % int par(int x) {
  %   if (x == 0) {
  %     return 0;
  %   } else if (x % 2 == 0) {
  %     return 1;
  %   } else {
  %     return -1;
  %   }
  % }

  \item (0,7 pt) Escreva a definição para a estrutura de um elemento de uma lista encadeada que armazena uma string de 50 caracteres no seu campo de dados. Defina também um sinônimo para o nome dessa estrutura.

  \vfill\null
  \columnbreak

  \item (1,0 pt) Escreva as linhas que o programa a seguir imprime na tela. Não esqueça de levar em conta quando uma linha é pulada (\texttt{\textbackslash n}). \textbf{Atenção}: não descreva o que vai ser escrito ("o programa vai imprimir isso e aquilo"), escreva \textbf{literalmente} as linhas que vão aparecer na tela quando o programa for executado.
  % 1
  % 4
  % 9
  % 16
  % 25
  % 36
  % 49
  % 64
  % 81
  % 100
  % O total é 385

  \begin{minted}{c}
#include <stdio.h>

int main() {
  int x = 1;
  int total = 0;
  int y;

  while (x <= 10) {
    y = x * x;

    printf("%d\n", y);
    total += y;
    x++;
  }

  printf("O total é %d\n", total);
  return 0;
}
  \end{minted}

  \item (1,0 pt) Complete o quadro abaixo com o código necessário para completar a tarefa a seguir: abrir um arquivo chamado \texttt{"arquivo.txt"} em uma variável chamada \texttt{arq}, no modo de leitura e escrita, de forma que os dados sejam adicionados ao final do arquivo. Escrever a string \texttt{sigla\_ufal} no arquivo.
  % FILE* arq;
  %
  % arq = fopen("arquivo.txt", "a+");
  % fputs(sigla_ufal, arq);

  \begin{minted}{c}
#include <stdio.h>

int main () {
  char sigla_ufal[5] = "UFAL";
  // << SEU CÓDIGO ENTRA AQUI >>
  fclose(arq);
}
  \end{minted}

\end{enumerate}
\end{multicols*}
\end{document}
