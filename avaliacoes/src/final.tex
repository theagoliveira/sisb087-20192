\documentclass[a4paper,10pt]{article}

\usepackage{appendixnumberbeamer}
\usepackage{booktabs}
\usepackage[scale=2]{ccicons}
\usepackage{pgfplots}
\usepackage{xspace}
\usepackage{bookmark}
\usepackage{amssymb}
\usepackage{mathtools}
\usepackage[normalem]{ulem}
\usepackage[T1]{fontenc}
\usepackage[sfdefault,book]{FiraSans}
\usepackage{FiraMono}
\usepackage{fontawesome}
\usepackage{hyperref}
\usepackage{listings}

\renewcommand{\thefootnote}{\fnsymbol{footnote}}
\renewcommand{\MintedPygmentize}{/home/thiago/.local/bin/pygmentize}

\definecolor{light-gray}{gray}{0.95}
\renewcommand\lstlistingname{Código}
\DeclareCaptionFormat{listing} {
  \parbox{\textwidth}{\hspace{-0.2cm}#1#2#3}
}
\DeclareCaptionFont{black}{\color{black}}
\captionsetup[lstlisting]{
  format=listing,
  labelfont=black,
  textfont=black,
  singlelinecheck=true,
  margin=0pt,
  font={tt,footnotesize,bf}
}

\lstset{
  basicstyle=\footnotesize\ttfamily,
  escapeinside={\%*}{*)},
  mathescape=true,
  showspaces=false,
  showtabs=false,
  showstringspaces=false,%
  rulesepcolor=\color{black},
  frame=shadowbox,
  upquote=true,
  literate=
  {á}{{\'a}}1 {é}{{\'e}}1 {í}{{\'i}}1 {ó}{{\'o}}1 {ú}{{\'u}}1
  {Á}{{\'A}}1 {É}{{\'E}}1 {Í}{{\'I}}1 {Ó}{{\'O}}1 {Ú}{{\'U}}1
  {à}{{\`a}}1 {è}{{\`e}}1 {ì}{{\`i}}1 {ò}{{\`o}}1 {ù}{{\`u}}1
  {À}{{\`A}}1 {È}{{\'E}}1 {Ì}{{\`I}}1 {Ò}{{\`O}}1 {Ù}{{\`U}}1
  {ä}{{\"a}}1 {ë}{{\"e}}1 {ï}{{\"i}}1 {ö}{{\"o}}1 {ü}{{\"u}}1
  {Ä}{{\"A}}1 {Ë}{{\"E}}1 {Ï}{{\"I}}1 {Ö}{{\"O}}1 {Ü}{{\"U}}1
  {â}{{\^a}}1 {ê}{{\^e}}1 {î}{{\^i}}1 {ô}{{\^o}}1 {û}{{\^u}}1
  {Â}{{\^A}}1 {Ê}{{\^E}}1 {Î}{{\^I}}1 {Ô}{{\^O}}1 {Û}{{\^U}}1
  {Ã}{{\~A}}1 {ã}{{\~a}}1 {Õ}{{\~O}}1 {õ}{{\~o}}1
  {œ}{{\oe}}1 {Œ}{{\OE}}1 {æ}{{\ae}}1 {Æ}{{\AE}}1 {ß}{{\ss}}1
  {ű}{{\H{u}}}1 {Ű}{{\H{U}}}1 {ő}{{\H{o}}}1 {Ő}{{\H{O}}}1
  {ç}{{\c c}}1 {Ç}{{\c C}}1 {ø}{{\o}}1 {å}{{\r a}}1 {Å}{{\r A}}1
  {€}{{\euro}}1 {£}{{\pounds}}1 {«}{{\guillemotleft}}1
  {»}{{\guillemotright}}1 {ñ}{{\~n}}1 {Ñ}{{\~N}}1 {¿}{{?`}}1
}

\setminted{
  fontsize=\footnotesize,
  style=bw,
  frame=single,
  labelposition=topline,
}

\pretitle{\begin{center}\normalsize\bfseries\MakeUppercase{Programação 2 -- }\MakeUppercase}
\author{\normalsize Prof. Thiago Cavalcante}
\date{\vspace{-4ex}}

\geometry{
  top=0.0cm,
  left=1.0cm,
  right=1.0cm,
  bottom=1.0cm
}

\setlength\columnsep{30pt}

\usetikzlibrary{arrows}

\makeatletter
\AddEnumerateCounter{\PaddingUp}{\two@digits}{A00}
\AddEnumerateCounter{\PaddingDown}{\two@digits}{A00}
\newcommand\PaddingUp[1]{\expandafter\two@digits\csname c@#1\endcsname}
\newcommand\PaddingDown[1]{\PaddingUp{#1}\addtocounter{#1}{-2}}
\makeatother

\makeatletter
\def\enumalphalphcnt#1{\expandafter\@enumalphalphcnt\csname c@#1\endcsname}
\def\@enumalphalphcnt#1{\AlphAlph{#1}}
\makeatother
\AddEnumerateCounter{\enumalphalphcnt}{\@enumalphalphcnt}{aa}

\pagenumbering{gobble}


\title{Final}

\begin{document}

\maketitle

\emergencystretch 3em

\begin{itemize}[itemsep=0em]
  \item Não use celular/computador e não converse com ninguém, a prova é individual.
  \item Sinta-se à vontade para tirar dúvidas (\textbf{razoáveis}) ou pedir esclarecimentos sobre as questões.
  \item Use \textbf{letra legível}! não posso dar nota para algo que não consigo ler.
  \item Lembre-se de \textbf{assinar seu nome nas suas folhas}. Se usar \textbf{mais de uma} folha, \textbf{enumere cada página}.
  \item \textbf{Seja organizado:} especifique número e letra da questão que você está respondendo e deixe um espaço entre as respostas, para não ficar tudo amontoado. Você pode pegar mais folhas, se precisar.
\end{itemize}


NOME: \rule{.85\textwidth}{0.1mm}

\begin{multicols*}{2}
\setlength{\leftmargini}{0pt}
\begin{enumerate}
  \item (3,5 pt $\rightarrow$ 7 x 0,5 pt) Preencha os espaços a seguir.

  \begin{enumerate}
    \item O comando \rule{1cm}{0.2mm} em uma função é usado para passar o valor de uma expressão de volta ao ponto do programa onde a função foi chamada. % return
    \item A função \rule{1cm}{0.2mm} normalmente é usada quando se lê blocos de bytes de um arquivo binário. % fread
    \item Uma função que chama a si mesma é uma função \rule{1cm}{0.2mm}. % recursiva
    \item Os nomes dos cinco primeiros elementos de um array \texttt{p} são \rule{1cm}{0.2mm}, \rule{1cm}{0.2mm}, \rule{1cm}{0.2mm}, \rule{1cm}{0.2mm} e \rule{1cm}{0.2mm}. % p[0], p[1], p[2], p[3], p[4]
    \item A função \rule{1cm}{0.2mm} escreve uma string em um arquivo. % fputs
    \item Um elemento de uma lista encadeada guarda informações e um \rule{1cm}{0.2mm} para o próximo elemento da lista. % ponteiro
    \item Uma variável de ponteiro contém como valor o \rule{1cm}{0.2mm} de outra variável. % endereço
  \end{enumerate}

  \item (2,8 pt $\rightarrow$ 4 x 0,7 pt) Verdadeiro ou falso. Justifique sua resposta para afirmações falsas.

  \begin{enumerate}
    \item O operador \texttt{\&} retorna o local na memória em que seu operando está armazenado % Verdadeiro.
    \item A função \texttt{fprintf} não pode imprimir os dados na tela. % Falso. Pode ser usada para ler dados do teclado usando o ponteiro para o arquivo stdout.
    \item Um array pode armazenar muitos tipos diferentes de valores. % Falso. Um array armazena apenas um tipo de valor.
    \item Os campos de diferentes estruturas podem ter nomes iguais. % Verdadeiro.
  \end{enumerate}

  \item (1,0 pt) Escreva uma função que recebe um número inteiro e retorna:

  \begin{itemize}
    \item \textcolor{white}{-}0, se ele for igual a zero
    \item \textcolor{white}{-}1, se ele for par
    \item -1, se ele for ímpar
  \end{itemize}
  % int par(int x) {
  %   if (x == 0) {
  %     return 0;
  %   } else if (x % 2 == 0) {
  %     return 1;
  %   } else {
  %     return -1;
  %   }
  % }

  \item (0,7 pt) Escreva a definição para a estrutura de um elemento de uma lista encadeada que armazena uma string de 50 caracteres no seu campo de dados. Defina também um sinônimo para o nome dessa estrutura.

  \vfill\null
  \columnbreak

  \item (1,0 pt) Escreva as linhas que o programa a seguir imprime na tela. Não esqueça de levar em conta quando uma linha é pulada (\texttt{\textbackslash n}). \textbf{Atenção}: não descreva o que vai ser escrito ("o programa vai imprimir isso e aquilo"), escreva \textbf{literalmente} as linhas que vão aparecer na tela quando o programa for executado.
  % 1
  % 4
  % 9
  % 16
  % 25
  % 36
  % 49
  % 64
  % 81
  % 100
  % O total é 385

  \begin{minted}{c}
#include <stdio.h>

int main() {
  int x = 1;
  int total = 0;
  int y;

  while (x <= 10) {
    y = x * x;

    printf("%d\n", y);
    total += y;
    x++;
  }

  printf("O total é %d\n", total);
  return 0;
}
  \end{minted}

  \item (1,0 pt) Complete o quadro abaixo com o código necessário para completar a tarefa a seguir: abrir um arquivo chamado \texttt{"arquivo.txt"} em uma variável chamada \texttt{arq}, no modo de leitura e escrita, de forma que os dados sejam adicionados ao final do arquivo. Escrever a string \texttt{sigla\_ufal} no arquivo.
  % FILE* arq;
  %
  % arq = fopen("arquivo.txt", "a+");
  % fputs(sigla_ufal, arq);

  \begin{minted}{c}
#include <stdio.h>

int main () {
  char sigla_ufal[5] = "UFAL";
  // << SEU CÓDIGO ENTRA AQUI >>
  fclose(arq);
}
  \end{minted}

\end{enumerate}
\end{multicols*}
\end{document}
