\documentclass[a4paper,10pt]{article}

\usepackage[utf8]{inputenc}
\usepackage[T1]{fontenc}
\usepackage{enumitem}
\usepackage{amssymb}
\usepackage{amsmath}
\usepackage{fontawesome}
\usepackage[top=3cm,left=44mm,right=44mm,bottom=2cm]{geometry}
\usepackage[brazil]{babel}
\usepackage{sourcecodepro}

\input{configs.tex}

\title{Reavaliação AB1}

\begin{document}

\maketitle

\emergencystretch 3em

\input{recommendations.tex}

NOME: \rule{.85\textwidth}{0.1mm}

\begin{multicols*}{2}
\setlength{\leftmargini}{0pt}
\begin{enumerate}
  \item (3,0 pt $\rightarrow$ 6 x 0,5 pt) Preencha os espaços a seguir.

  \begin{enumerate}
    \item O comando \rule{1cm}{0.2mm} em uma função é usado para passar o valor de uma expressão de volta ao ponto do programa onde a função foi chamada. % return
    \item Listas e tabelas de valores são armazenadas em \rule{1cm}{0.2mm}. % arrays
    \item Uma matriz m-por-n contém \rule{1cm}{0.2mm} linhas, \rule{1cm}{0.2mm} colunas e \rule{1cm}{0.2mm} elementos. % m, n, m x n
    \item Uma variável de ponteiro contém como valor o \rule{1cm}{0.2mm} de outra variável. % endereço
    \item A função \rule{1cm}{0.2mm} exibe dados formatados na tela. % printf
    \item A palavra-chave \rule{1cm}{0.2mm} introduz uma declaração de uma estrutura. % struct
  \end{enumerate}

  \item (2,1 pt $\rightarrow$ 3 x 0,7 pt) Verdadeiro ou falso. Justifique sua resposta para afirmações falsas.

  \begin{enumerate}
    \item Um elemento de uma lista que é passado para uma função na forma \texttt{a[i]} e é modificado dentro dessa função terá seu valor modificado também fora da função. % Falso. O elemento é passado por valor e isso cria uma cópia independente na memória.
    \item Uma definição de array reserva espaço na memória para o array. % Verdadeiro.
    \item Os campos de diferentes estruturas podem ter nomes iguais. % Verdadeiro.
  \end{enumerate}

  \item (1,5 pt) O que o programa a seguir imprime na tela? Não esqueça de levar em conta quando uma linha é pulada (\texttt{\textbackslash n}).
  % 1
  % 4
  % 9
  % 16
  % 25
  % 36
  % 49
  % 64
  % 81
  % 100
  % O total é 385

  \begin{minted}{c}
#include <stdio.h>

int main() {
  int x = 1, total = 0, y;

  while (x <= 10) {
    y = x * x;

    printf("%d\n", y);
    total += y;
    x++;
  }

  printf("O total é %d\n", total);
  return 0;
}
  \end{minted}

  \vfill\null
  \columnbreak

  \item (1,0 pt) Escreva uma função que retorne o menor de três números decimais. Retorne -1 se os três números forem iguais.
  % int minimo(int a, int b, int c) {
  %   if (a == b) && (b == c)
  %     return -1;
  %   if (a <= b) && (a <= c)
  %     return a;
  %   if (b <= a) && (b <= c)
  %     return b;
  %   if (c <= b) && (c <= a)
  %     return c;
  % }

  \item (1,0 pt) Escreva uma função que recebe um par de inteiros e determina se o segundo inteiro é um múltiplo do primeiro.
  % int multiplo(int a, int b) {
  %   return (b % a == 0);
  % }

  \item (1,4 pt) O código a seguir contém uma função recursiva que recebe um par de inteiros, \texttt{base} e \texttt{exp}, e retorna $\text{base}^\text{exp}$. Complete o critério de parada e a chamada recursiva da função.
  % exp == 0
  % base, exp - 1

  \begin{minted}{c}
int exp_rec(int base, int exp) {
  if (/* CRITÉRIO DE PARADA */) {
    return 1;
  } else {
    return base * exp_rec(/* CHAMADA RECURSIVA */);
  }
}
  \end{minted}
\end{enumerate}
\end{multicols*}

\pagebreak

\maketitle

\input{recommendations.tex}

NOME: \rule{.85\textwidth}{0.1mm}

\begin{multicols*}{2}
\setlength{\leftmargini}{0pt}
\begin{enumerate}
  \item (3,0 pt $\rightarrow$ 6 x 0,5 pt) Preencha os espaços a seguir.

  \begin{enumerate}
    \item A palavra-chave \rule{1cm}{0.2mm} é usada no cabeçalho de uma função para indicar que ela não retorna um valor ou para indicar que uma função não contém parâmetros. % void
    \item Os elementos de um array têm em comum o fato de que possuem o mesmo \rule{1cm}{0.2mm}. % tipo
    \item O nome do elemento de um array \texttt{d} na linha 3, coluna 5 é \rule{1cm}{0.2mm}. % d[3][5]
    \item O operador \rule{1cm}{0.2mm} retorna o local na memória em que seu operando está armazenado. % &
    \item Todas as strings terminam no caractere \rule{1cm}{0.2mm}. % \0
    \item A palavra-chave \rule{1cm}{0.2mm} é usada para criar um sinônimo para um tipo de dado previamente definido. % typedef
  \end{enumerate}

  \item (2,1 pt $\rightarrow$ 3 x 0,7 pt) Verdadeiro ou falso. Justifique sua resposta para afirmações falsas.

  \begin{enumerate}
    \item Um índice de array pode ser do tipo float. % Falso. O índice precisa ser um número inteiro.
    \item Para indicar que 100 locais devem ser reservados para o array de inteiros \texttt{p}, escreva a declaração \texttt{int p[100];} % Verdadeiro.
    \item Para acessar um elemento em um array, especificamos o nome do array e o índice do elemento em particular. % Verdadeiro.
  \end{enumerate}

  \item (1,5 pt) O que o programa a seguir imprime na tela? Não esqueça de levar em conta quando uma linha é pulada (\texttt{\textbackslash n}).
  % ****
  % ++++++++
  % ****
  % ++++++++
  % ****
  % ++++++++

  \begin{minted}{c}
#include <stdio.h>
int main() {
  int contador = 1;
  char simbolos_a[10] = "****";
  char simbolos_b[10] = "++++++++";

  while (contador <= 6) {
    if (contador % 2 == 1) {
      printf("%s\n", simbolos_a);
    } else {
      printf("%s\n", simbolos_b);
    }
    contador++;
  }

  return 0;
}
  \end{minted}

  \vfill\null
  \columnbreak

  \item (1,0 pt) Escreva uma função que recebe um intervalo de tempo como três argumentos inteiros (horas, minutos e segundos) e retorna esse intervalo em segundos.
  % int segundos(int h, int m, int s) {
  %   return (s + (60 * m) + (60 * 60 * h));
  % }

  \item (1,0 pt) Escreva uma função que recebe um par de números inteiros, \texttt{base} e \texttt{exp}, e retorna $\text{base}^\text{exp}$. Não use funções da biblioteca de matemática.
  % int expoente(int base, int exp) {
  %   int contador, resultado;
  %   contador = 0;
  %   resultado = 1;
  %
  %   while(contador < exp) {
  %     resultado *= base;
  %     contador++;
  %   }

  \item (1,4 pt) O código a seguir contém uma função recursiva que recebe um array de inteiros e seu tamanho e retorna seu valor máximo. Complete o critério de parada e a chamada recursiva da função.
  % tam == 1
  % &array[1], tam - 1

  \begin{minted}{c}
int max_rec(int *array, int tam) {
  int max;

  if (/* CRITÉRIO DE PARADA */) {
    return array[0];
  } else {
    max = max_rec(/* CHAMADA RECURSIVA */);
    if (array[0] > max) {
      return array[0];
    } else {
      return max;
    }
  }
}
  \end{minted}
\end{enumerate}
\end{multicols*}

\pagebreak

\maketitle

\input{recommendations.tex}

NOME: \rule{.85\textwidth}{0.1mm}

\begin{multicols*}{2}
\setlength{\leftmargini}{0pt}
\begin{enumerate}
  \item (3,0 pt $\rightarrow$ 6 x 0,5 pt) Preencha os espaços a seguir.

  \begin{enumerate}
    \item Uma função que chama a si mesma é uma função \rule{1cm}{0.2mm}. % recursiva
    \item O número de referência a um elemento em particular de um array é chamado seu \rule{1cm}{0.2mm}. % índice
    \item Os nomes dos cinco primeiros elementos de um array \texttt{p} são \rule{1cm}{0.2mm}, \rule{1cm}{0.2mm}, \rule{1cm}{0.2mm}, \rule{1cm}{0.2mm} e \rule{1cm}{0.2mm}. % p[0], p[1], p[2], p[3], p[4]
    \item O operador \rule{1cm}{0.2mm} retorna o valor do objeto ao qual seu operando aponta. % *
    \item A função \rule{1cm}{0.2mm} lê dados formatados do teclado. % scanf
    \item O operador \rule{1cm}{0.2mm} é usado para acessar os campos de um ponteiro para uma \texttt{struct}. % ->
  \end{enumerate}

  \item (2,1 pt $\rightarrow$ 3 x 0,7 pt) Verdadeiro ou falso. Justifique sua resposta para afirmações falsas.

  \begin{enumerate}
    \item Um array pode armazenar muitos tipos diferentes de valores. % Falso. Um array armazena apenas um tipo de valor.
    \item O operador \texttt{\&} retorna o local na memória em que seu operando está armazenado % Verdadeiro.
    \item O nome de uma estrutura \texttt{struct nome \{ ... \};} é opcional. % Verdadeiro.
  \end{enumerate}

  \item (1,5 pt) O que o programa a seguir imprime na tela? Não esqueça de levar em conta quando uma linha é pulada (\texttt{\textbackslash n}).
  % >>>>>>
  % <<<<<<
  % >>>>>>
  % <<<<<<
  % >>>>>>
  % <<<<<<

  \begin{minted}{c}
#include <stdio.h>

int main() {
  int linha = 6;
  int coluna;
  char simbolo;

  while (linha >= 1) {
    coluna = 1;
    while (coluna <= 6) {
      if (linha % 2 == 1) {
        simbolo = '<';
      } else {
        simbolo = '>';
      }
      printf("%d", simbolo);
      coluna++;
    }
    linha--;
    printf("\n");
  }

  return 0;
}
  \end{minted}

  \vfill\null
  \columnbreak

  \item (1,0 pt) Escreva uma função que recebe um número inteiro e retorna: 0, se ele for igual a zero; 1, se ele for par; -1, se ele for ímpar.
  % int par(int x) {
  %   if (x == 0) {
  %     return 0;
  %   } else if (x % 2 == 0) {
  %     return 1;
  %   } else {
  %     return -1;
  %   }
  % }

  \item (1,0 pt) Escreva uma função que recebe um array de inteiros e seu tamanho e retorna a média dos elementos.
  % float media(int *array, int tam) {
  %    float resultado;
  %    int i;
  %
  %    for(i = 0; i < tam; i++)
  %       resultado += array[i];
  %
  %    return (float) resultado/tam;

  \item (1,4 pt) O código a seguir contém uma função recursiva que recebe um array de inteiros e seu tamanho e imprime todos os seus elementos em ordem inversa. Complete o critério de parada e a chamada recursiva da função.
  % tam == 1
  % &array[1], tam - 1

  \begin{minted}{c}
void impr_rec(int *array, int tam) {
  if (/* CRITÉRIO DE PARADA */) {
    printf("%d ", array[0]);
  } else {
    impr_rec(/* CHAMADA RECURSIVA */);
    printf("%d ", array[0]);
  }
}
  \end{minted}
\end{enumerate}
\end{multicols*}
\end{document}
